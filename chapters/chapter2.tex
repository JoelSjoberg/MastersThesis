In this chapter we review the concepts on which this thesis is based. We first cover the statistical theory for understanding the methods of analysis used in Chapter 3 and use of Machine Learning (ML) in Chapter 4. We then proceed by introducing statistical concepts required for understanding the methods employed in the project. We then review common concepts and specific methods within machine learning which is the central theme in this thesis.


\section{Statistics}

In this section we discuss the primary statistical concepts required for understanding the methods used in this thesis. The first concepts necessary for the later chapters are the mean and the standard deviation. The mean is a value used to describe the average value of a population. A population is the term used to describe the complete collection of elements in some context e.g. all people alive, all numbers in an interval etc. The mean is an important concept in statistics as it is often used as a characteristic of the elements found in the population under analysis. The mean is calculated by the sum of the each element in the population divided by the number of elements in said population. The mean $\mu$ of a population of $n$ numbers $L$, is calculated as expressed in equation (\ref{eqn:mean}).

\begin{equation}
\label{eqn:mean}
\mu = \sum_{i=1}^n \frac{L_i}{n}
\end{equation}

The mean of a population is usually used in association with the standard deviation. The standard deviation is the square root of the variance of that population. The variance is an expression for the scaled summed squares of differences from the mean of the entire population. Calculating the square root then produces a value expressing the the dispersion of elements within the population around the mean. The variance is calculated by measuring the summed squared distance between each element within the population and the mean, divided by the number of elements in the population to scale the sum. In the case where the entire population is impossible to analyze or unknown, samples are extracted from the population (i.e. subgroups of elements randomly taken from the population). The sample variance changes slightly form the population variance; it divides the summed squares of differences by the number of elements in the sample subtracted by one. Subtracting the original denominator by one is called "Bessel's correction". The correction is based on the fact that, if the population mean is unknown (as is often the case when samples are gathered), then the mean used in calculating the variance will only be an estimation. Subtraction by one is performed to avoid bias towards the sample mean and get an unbiased estimation of the population variance \cite{so2008sample, nobach2020practical}. The standard deviation $\sigma$ of a sample is thus calculated as expressed in equation (\ref{eqn:std}).

\begin{equation}
\label{eqn:std}
 \sigma = \sqrt{\frac{\sum_{i = 1}^{n} (\mu - L_i)^2}{n-1}}
\end{equation}


\subsection{The standard deviation test}
The mean and standard deviation can be used to detect outliers in a sample of data points drawn from an unknown population. The Gaussian distribution (also called the normal distribution) is used in association with the mean and the standard deviation. If the frequency which elements appear within a sample are more likely to be close to the mean than far away from it, we say that the elements within that sample are normally distributed. If an element differs from the mean by more than three standard deviations, the possibility of that element not belonging to that distribution is extremely high. Such elements, which likely do not belong to the population, are called outliers. The method which detects outliers by measuring the distance between each element and the mean is referred to, in this thesis, as the standard deviation test (SDT). The test assumes that elements within the collection are normally distributed with some mean $\mu_c$ and some standard deviation $\sigma_c$. By computing the \textit{z-score} of the elements, the data is transformed into a standardized form through standardization. \textit{Z-score} standardization subtracts the mean from all elements in the collection and divides the difference by the standard deviation, as expressed in equation (\ref{eqn:zscore}) \cite{geary1935ratio}.

\begin{equation}
\label{eqn:zscore}
 \frac{L_i - \mu_c} {\sigma_c}
\end{equation}

Following standardization, the entire sample will have a mean of zero and a standard deviation of one. Each element in the sample can then be measured using the standard deviation (one) as unit; outliers can then be discarded from the sample by removing elements which have an absolute value of 3 or higher.


\subsection{The interquartile range method}
An alternative method suited for outlier detection is the interquartile range method (IRM). IRM is based on analyzing the sample by its median, which is the center of the sorted sample. It is applied by sorting the sample elements in ascending order and organizing the sample into four percentiles, each percentile distanced from the others by $25\%$ of the sample.  Two quartiles center around $50\%$ of the sample contents, this center is referred to as the interquartile range. IRM requires two values from the sample, those are the highest value of the $25$th percentile and the highest value of the $75$th percentile. The first value is gained by taking the biggest value of the $q_{25}$ first elements from the sorted collection, where $q_{25}$ is $25\%$ of the number of elements in $L$ (calculated by $0.25 \cdot |L|$). The last percentile is gained in similar fashion, with the exception that the highest value is taken from the first $75\%$ of the sorted data. Two "cut-off" points are then defined, by multiplying each of the two values by a constant $k$ (which is called the "cut-off" constant). Elements can then be labeled as outliers if those elements fall below the lower "cut-off" point or above the higher "cut-off" point. This method is well suited for sample sizes which contain approximately $100$ elements but may prove cumbersome when the number of elements exceed that value \cite{vinutha2018detection, walfish2006review, dovoedo2015boxplot}.


\subsection{Analysis of variance}

The analysis of variance (ANOVA) is a method for statistical analysis developed by Ronald Fisher to analyze the difference among sample means in a collection of samples. It is based on the null-hypothesis stating the sample means of two or more samples are the same. The analysis then yields a \textit{F-value} and a \textit{p-value} which are used with the F-distribution to accept or reject the null-hypothesis. The analysis assumes the population the samples are drawn from in normally distributed and that the samples are independent of each other. It also assumes the standard deviations and variances are roughly equal among the samples.

The analysis is performed by calculating the mean of each sample. The summed square of differences from each mean are then calculated for their respective sample, the result is then subtracted by the squared sum of the elements within the sample divided by the number of elements in said sample. This calculation for sample $S$ is formally expressed i equation (\ref{eqn:SS}).

\begin{equation}
\label{eqn:SS}
 \sum_s^S (s - \mu_S)^2 - \frac{(\sum_{s}^{S} s)^2}{|S|} 
\end{equation}

This is performed once for all samples drawn from the population, the results are then summed together into a sum of sample sums (SS). The calculation is also performed once on the total collection of all elements from all samples, the result is stored in the total sum of sample sums (SST). The total sum of squares is the sum of SS and the sum of squared distances from each sample mean to the total mean (SSM). SSM is therefore calculated by subtracting SST and SS. The analysis also require divisions by two values called the degrees of freedom. They are calculated as follows: $d_1$ is the number of samples subtracted by one and $d_2$ is the number of elements in the total collection subtracted by the number of samples. The \textit{F-value} is calculated by a fraction of two fractions. Let  $K_1$ be the fraction $\frac{SST}{d_1}$ and $K_2$ be the fraction $\frac{SS}{d_2}$, the \textit{F-value} is defined as expressed in equation(\ref{eqn:Fval}).

\begin{equation}
\label{eqn:Fval}
 F = \frac{\frac{SST}{d_1}}{\frac{SS}{d_2}} = \frac{K_1}{K_2}
\end{equation}

The quotient yielded by the fraction in equation (\ref{eqn:Fval}) can then be inserted into a pre-calculated table of the F-distribution provided the degrees of freedom to yield the \textit{p-value}. A small \textit{p-value} (i.e. lower than $0.05$) indicates that the null-hypothesis may be rejected with relative certainty whereas a high \textit{p-value} indicates the null-hypothesis holds and should be accepted with relative certainty \cite{lowry2014concepts}.


\section{Machine Learning}

Machine learning is the practice of computing models for relationships between sets of data. The field has garnered significant interest within academia and industry alike due to the promising results in applications for which deterministic algorithms have proven difficult or impossible to make. Examples of such applications are computer vision, natural language processing and personalized advertising, to name a few \cite{sebe2005machine, allen2003natural, malheiros2012too}. There are two main paradigms for learning: Supervised learning (using labeled data to approximate models) and unsupervised learning (finding patterns within the data itself). 

Models are used to great length within many scientific domains. In the context of machine learning, a model can be seen as a data structure made out of constant parameters combined with an algorithm which utilizes the data structure to produce predictions given an input vector (the input can also be in the form of a multi-dimensional matrix). 

The model can be represented mathematically as a collection of structures in the form of vectors or matrices, the elements of which are referred to as parameters. A model can consist of learnable parameters $\theta$ and non-learnable parameters (often generated by stochastic initialization). The model computes a function $f$ to yield a prediction $y$ by applying the algorithm to the parameters given an input example $x$ (which can be a vector or a matrix) drawn form the data set $X$. An example of this process, where the algorithm produces predictions through addition, is given in equation (\ref{eqn:simplemodel}).

\begin{equation}
\label{eqn:simplemodel}
 y = x_0\theta_0 + x_1\theta_1 + ... + x_n\theta_n
\end{equation}

Machine learning then, is the practice of changing (also known as tuning) $\theta$ by introducing small changes to the elements within $\theta$. This is done to minimize a loss function $L$ which computes the error (or loss) given $y$. The process of changing $\theta$ is known as the training process and is central to machine learning. In the training process for supervised learning, the data gathered for the model is separated into three sets. These sets are referred to as the training set, validation set and test set. They are randomly collected samples from the common data set such that the intersection between the sets is empty. The purpose of the training process is to train the model on the training data and use the validation and test sets as a means to validate the model performance on data not encountered during the training process. Supervised learning requires that the examples used have a label which the model tries to predict (data which possess labels are called labeled data). We say that a model generalizes well to the data if the training process allows the model to perform well on unseen data. If the model manages to perform well on the training set but fails to generalize, the model is said to overfit to the data. Unsupervised learning is a learning paradigm which does not rely on the use of labeled data. Instead, the paradigmn focuses on organizing the data in a way that minimizes $L$. Predictions can then be performed by evaluating the way the data has been organized by some method related to the problem context.


\subsection{K-means Clustering}

Clustering is an unsupervised learning method whose primary use is in grouping sets of data. In this thesis we consider the \textit{K-means clustering} algorithm. The following is a formal definition of \textit{K-means clustering} as defined by MacQueen \cite{macqueen}. Given a set of $N$-dimensional points (where $N \in \mathbb{N}$) $E_N$ and a desired amount of partitions $k$ of $E_N$, partition the elements of $E_N$ into $k$ sets, the subsets are stored in a superset $S$ such that $S = \{S_1, S_2, ... S_k\}$. The partitioning of $E_N$ is performed by randomly initializing $k$  $N$-dimensional points as randomly selected points within $E_N$. We define the set $V$ with elements $v$ where $v_i$ is the i:th cluster center where $i \in [0, k]$. The partitioning of the elements $x \in E_N$ into their respective partition $S_i$ is performed by computing the closest cluster center $\forall_{x \in E_N}$. Let $T_i$ be the set of $x \in E_N$ such that the distance from the element to the relevant cluster is minimal, $T_i$ is defined by formula (\ref{eqn:Ti}).

\begin{equation}
\label{eqn:Ti}
T_i = \{x : x \in E_N | (|x - v_i| \leq |x - v_j|)\}  0 \leq j \leq k \wedge j \in \mathbb{N} 
\end{equation}

For centers who share equal distance to any given $x$ the cluster with the smallest index is chosen as the containing set. Let $S_i^c$ denote the complement of set $S_i$, then the partitions $S_i \in S$ are defined by formula (\ref{eqn:Si}).

\begin{equation}
\label{eqn:Si}
S_i = T_i \cap \bigcap_{j=1}^{(i-1)} S_j^c
\end{equation}

A consequence to this definition is that outliers have a potential to drastically change the quality of the cluster outcomes \cite{chawla2013k}. To remedy this and the stochastic nature of the initialization process, the method is run several times on the same dataset, yielding the resulting clusters with minimal inertia. The problem \textit{K-means clustering} attempts to solve is proven to be NP-hard \cite{chawla2013k, mahajan2009planar} but the algorithm itself has a time complexity of $O(n^2)$ \cite{pakhira2014linear}.


\subsection{Hierarchical clustering}

Hierarchical clustering is a clustering method which has a deterministic process. Each cluster formed is based on the entire dataset in contrast to \textit{K-means} which approximates clusters by performing small changes to the cluster centers. The method produces clusters by iteratively combining the closest clusters according to the given linkage criterion (defined in the sections below). The two primary strategies for forming clusters are \textit{agglomerative} and \textit{divisive}. Agglomerative clustering initializes one cluster for each data point and combines them in a hierarchy according to the linkage criterion until all clusters are part of the hierarchy. Divisive strategies initializes one universal cluster for all data points and proceeds to separate the points into distinct clusters according to the linkage criterion. The method proceeds until all data points are separated to their own cluster within the unifying hierarchy. The project described in this thesis uses the \textit{agglomerative} strategy. All strategies depend on the distance measure and linkage criterion \cite{murtagh1983survey}.

\subsubsection{Distance metrics}
Let $u$ and $v$ be vectors of the same dimension $n$. The \textit{Euclidean distance} (also called \textit{L2-distance}) measure can be used to measure distance between the vectors in euclidean space. The \textit{Euclidean distance} is calculated by equation (\ref{eqn:euclid}).

\begin{equation}
\label{eqn:euclid}
d(u, v) = \sqrt{\Sigma_i (u_i - v_i)^2} 
\end{equation}

The \textit{Manhattan distance} (also called \textit{L1-distance}) metric is also a viable alternative if the distance is to be measured in blocks. The distance is akin to finding a shortest path among blocks and is therefore calculated as expressed in equation (\ref{eqn:manhattan}).

\begin{equation}
\label{eqn:manhattan}
d(u, v) = \Sigma_i |u_i - v_i|
\end{equation}

\textit{Cosine similarity} measures similarity between vector angles and suits situations where certain vectors are expected to be similar. Should the vectors be sizable in terms of dimensionality, this method will yield varying results, especially if the elements have significant variance in each dimension. It is calculated as expressed in equation (\ref{eqn:cosine}).

\begin{equation}
\label{eqn:cosine}
d(u, v) = \frac{\Sigma_i u_iv_i}{\sqrt{\Sigma_i u_i^2}\sqrt{\Sigma_i v_i^2}}
\end{equation}

\subsubsection{Linkage Criteria}
In order to measure distance between clusters it is essential to know between which points the distance should be measured, since clusters often consist of several points. Linkage criteria describes the method for determining how the distance metric will be applied. SKlearn define four criteria in the documentation \cite{scikit}: Single linkage, complete linkage, average linkage and ward linkage. Depending on which criterion is applied the results may differ considerably, it is therefore vital to have a formal understanding of their application and consequences.

Single linkage goes through each pair of clusters measuring the distance among all points within one with respect to the other. The distance between these clusters is determined to be the distance between the two closest points. Let $U$ be the elements in the first cluster and $V$ be the elements of the second. The distance between the first and the second cluster is defined formally in equation (\ref{eqn:single}).

\begin{equation}
\label{eqn:single}
d(U, V) = \forall_{u, v \in U, V} min(d(u, v))
\end{equation}

Single linkage tend to produce trivial results, forging a hierarchy in a chain where individual elements slowly merge with the bigger cluster. In contrast complete linkage considers the largest distance between two points for every pair of clusters. The distance between two clusters then become the distance between the points which are the furthest apart. Formally expressed in equation (\ref{eqn:complete}).

\begin{equation}
\label{eqn:complete}
d(U, V) = \forall_{u, v \in U, V} max(d(u, v))
\end{equation}

By considering the largest possible distance between two clusters it bypasses the setback by single linkage, allowing more clusters to form before merging into one unifying cluster. Average linkage calculates the average between all elements for every pair of clusters and merges the ones possessing to the minimal average distance. Formally described by equation (\ref{eqn:average}).

\begin{equation}
\label{eqn:average}
d(U, V) = \frac{1}{|U||V|}\sum\limits_u^U \sum\limits_v^V  d(U_u, V_v)
\end{equation}

Ward linkage represents distance by how much the summed square would increase by merging them. The method aims to merge the clusters such that the within cluster variance is minimal. Let $c_a$ be the center of cluster a, then ward linkage is expressed formally by equation (\ref{eqn:ward}) \cite{shalizi2009distances}.


\begin{equation}
\label{eqn:ward}
d(U, V) = \frac{|U||V|}{|U|-|V|}||c_U - c_V||^2
\end{equation}



\subsection{Feature Selection}

In many cases the data available contains numerous features, which often helps to building sufficient classifiers as the model may find non-trivial patterns among the features. To avoid expanding the dependence on large datasets, it is often necessary to strip the data of certain features which possess minimal correlation to other features or which lack that correlation entirely \cite{dash1997feature}. Features that possess the necessary expressive information are not always trivial, there are several ways in which they may be found. \textbf{TODO: Expand this, it's been 3 weeks already!}


\section{Deep Learning}

The field of Artificial Intelligence is founded on the notion of designing algorithms for solving problems. The field encountered tremendous progress in \textbf{[FIND YEAR, AI FOUNDATIONS]} referred to by \textbf{[NAME]} as the "look ma, no hands" era of Artificial Intelligence. One such method which have proven useful for these tasks is the practice of approximating models through Artificial Neural Networks.

\subsection{Artificial Neural Networks}
Artificial Neural Networks ("ANNs") have been used to great success during the 20th century [\textbf{Source Here}]. With the use of ANNs
several fields including Natural Language Processing, Encoding and Image classification have undergone revolutionary leaps in performance regarding optimization due to the predictive power of these networks [\textbf{Source Here}]. At the same time they are heavily criticized for their complexity, yielding a structure much more akin to a so called \textit{"black box"} than a reliable and deterministic method for prediction[\textbf{Source Here}]. This complexity is due to numerous different structural typologies available at present and an awesome number of tuned parameters which are modified with the goal of minimizing the predictive error [\textbf{Source Here}].

A consequence of this is hard skepticism in regards to the correctness of their functionality within practical use. While these models have shown great promise when compared to their human counterparts, the question remain whether or not perfect performance can be yielded from the constructed models.\\

\textbf{Definition 1.} Training an ANN is allowing minuscule changes through the randomly initialized structure in order to approximate a collection of nested functions $$f_n(f_{n-1}(...f_1(X)))$$ 

\subsection{Computational Representations}
The initial purpose of ANNs was to create a computational model of the human cortex which took the form of the McCulloch, Pitts neuron. The multilayer perceptron (MLP) introduced in \textbf{[year here]} formed the basic structure which would become ANNs. 



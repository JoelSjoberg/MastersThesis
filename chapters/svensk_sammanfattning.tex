\title{Svensk Sammanfattning}

Gliom är en typ av cancer som formas i hjärnan från glial cellerna. Tumörer som konsekvent formas av sjukdomens effekt kategoriseras av världs hälso organisationen (på engelska "\textit{World Health Organization}") genom olika grader vilka beskriver hastigheten med vilken cancern sprids, samt dess aggressivitet. Cancerns effekt på gliom-patienterna kan variera avsevärt mellan huvudvärk, illamående eller komplikationer gällande hjärnans funktion (t.ex. förvirring, problem med att prata och kommunicera, epileptiska anfall osv.). En patient som diagnoseras med gliom har ett fåtal år kvar att leva, det är ovanligt att patienter överlever i fem år \cite{glialcells, gallego2015nonsurgical, bleeker2012recent}. Diagnoserings processen includerar ett ingrepp, där sampel av tumören extraheras från patienten. Samplen analyseras sedan i labbet för att undersöka tumörens grad och bestämma den optimala medicinska kuren. Denna analys, följande samplets extrahering, kan räcka ett obestämt antal veckor. I denna avhandling beskrivs en analytisk process där vi analyserar Raman spektra extraherat från gliom tumörer, tagna från 45 patienter. Från vissa patienter har flera sampel av samma tumör tagits, vilket ger ett större antal sampel än patienter. Analysens syfte är att förbereda alla spektra för använding i maskininlärnings metoder. Detta innebär att alla sampel måste analyseras för att ta bort de spectra som inte kommer från tumör material. Det är sannolikt att en delmängd av alla spektra från ett givet sampel kan ha kommit från t.ex. blod, plast som reflekteras under tunna bitar av samplet, nekrotiska celler osv.

Raman spektroskopi används pga. den information som finns i varje spektra samt dess användning i tidigare projekt med maskininlärning \cite{ramanDL, ho2019rapid}. Ett spektrum består av $1738$ olika frekvenser; en del av analysen undersöker vilka av dessa frekvenser som bäst delar in alla spektra i olika kategorier. Kategorierna är baserade på olika sorter av muteringar i IDH generna. Dessa kategorier har identifierats av Ceccarelli et al. \cite{cellsubsets} och påstås vara ett bättre sätt att identifiera en bra medicinsk kur, än de grader som för tillfället används. De frekvenser som bäst delar alla spektra in i kategorierna beräknas före och sedan efter att spektra från icke-tumör material har separerats från tumör spektra.

\section*{Analys}

Analysen börjar med att avlägsna vissa sampel som visar sig vara problematiska att analysera. Två sampel avlegsnas, ett sampel från patient HF-1887 innehåller spektra som ligger på en kurvad grundlinje. Vilket innebär att värdet på frekvenserna stiger stegvis från den första frekvensen till den sista. Ett högt värde på en frekvens indikerar information från punkten där spektrumet är taget. Då grundlinjen inte är rak blir det omöjligt att jämföra olika frekvenser i spektrumet. Alla sampel från patient HF-3097 har en konsistent grundlinje med alla andra sampel, men det finns ett betydligt större antal av frekvenser med enorma värden jämfört med andra sampel. Vi väljer att avlägsna alla sampel från patient HF-3097, eftersom den enorma mängden höga frekvenser i alla spektra tyder på att spektrometern har gett felaktiga värden.

Vi fortsätter genom att balansera data mängden enligt kategorierna som vi vill separera alla spektra i. Det finns en stor variation mellan antalet spektra tillhörande varje klategori. Vi delar in alla sampel i olika data mängder som kan ska användas i maskininlärning. Dessa är tränings mängden, validerings mängden och test mängden. Eftersom det finns en risk för att tumörerna är heterogena, bestämmer vi att dela in dessa mängder enligt sampel, dvs. åtminstone ett unikt sampel från varje kategori finns i varje data mängd. Maskininlärnings modellerna som tränas på data mängden måste kunna lära sig dela in ett spektrum i den kategori spektrumet tillhör genom att träna på tränings mängden. Om detta steg ignoreras och olika spektra från samma sampel används för att träna och validera modellen, kommer resultaten inte vara trovärda om alla tumörer är heterogena. Efter mängd separationerna kan vi balancera tränings mängden genom att beräkna antalet spektra tillhörande varje kategori i mänden. Antalet spektra i tränings mängden tillhörande kategori $n$ kan då beskrivas genom $|LGm_n|$. Varje kategori balanceras genom att kopiera varje spektrum i en kategori $x$ gånger, då $x$ är resultatet av divisionen mellan antalet element i kategorin som utgör majoriteten i mängen och antalet spektra i kategori n. Detta beräknas enligt $x = \lfloor \frac{|LGm_m|}{|LGm_n|} \rfloor$ där $\lfloor$ och $\rfloor$ antyder avrundning neråt av kvoten. De frekvenser som bäst beskriver samplets kategori beräknas sedan genom användinng av f-classif metoden i SKlearn biblioteket i Python \cite{scikit}.

För att identifiera icke-tumör spektra testar vi olika metoder. Dessa metoder är standard avvikelse testet, kvartilavståndet och oövervakade inlärnings metoder som \textit{hierarchical clustering} och \textit{K-means clustering}. Dessa metoder jämförs med \textit{frekvens kriteriumet}, som är ett kriterium baserat på frekvenserna $1463$ till $1473$ på alla spektra. Om någon av dessa frekvenser är under $5000$ på ett spektrum, är det spektrumet från ett icke-tumör material. Standard avvikelse testet och kvartilavståndet ger linknande resultat på majoriteten av sampel. Av dessa två är det kvartilavståndet som ger resultat liknande vad \textit{frekvens kriteriumet} ger. \textit{K-means clustering} lyckas ge resultat som stämmer med \textit{frekvens kriteriumet}, algoritmen lyckas även hitta andra regioner på samplet som \textit{frekvens kriteriumet} inte hittar. Denna analys utförs genom att initializera sex modeller som delar in samplet i två, tre, fyra, fem, sex och sju kluster. Mellan alla dessa finns det inte ett klart bästa alternativ för att identifiera icke-tumör spektra i varje sampel. Med \textit{hierarchical clustering} algoritmen undersöker vi olika länknings kriterier som finns implementerade i SKlearn. Dessa är \textit{single linkage}, \textit{average linkage}, \textit{complete linkage} och \textit{ward linkage}. Av dessa kriterier, är det \textit{ward linkage} som producerar bäst resultat. Dessa resultat är jämförbara med \textit{frekvens kriteriumet} och \textit{K-means clustering}. Metoden är också konsistent och kan hitta alla icke-tumör spektra i majoriteten av sampel med tre kluster. I vissa sampel där antalet icke-tumör spektra är oansenligt behövs det fyra kluster eller mera.

\subsection*{Användning av Datan}

Vi väljer att separera icke-tumör spektra från tumör spektra genom att använda \textit{ward linkage} med \textit{hierarchical clustering} algoritmen. Alla spektra går nu igenom en förberedande process som maskininlärnings modeller kräver. Grundlinjen på alla spektra tas bort genom \textit{ZhangFit} metoden i Python. Alla frekvenser i varje spektra är sedan standardiserade genom att subtrahera varje enskilld frekvens med den frekvensens medeltal, differensen divideras sedan med standard avvikelsen för den frekvensen. Alla frekvenser normaliseras sedan genom att dividera varje enskilld frekvens i varje spektra med det maximala absolutbelopp för den frekvensen. För att testa datans användbarhet efter den förberedande processen, skapar vi ett djupt neuralt nätverk för att klassificera alla spektra enligt deras respektive kategorier Ceccarelli et al. \cite{cellsubsets} presenterat. Modellens arkitektur beskrivs och dess prestanda presenteras. 

\subsection*{Slutsats}

Flera test på olika modell arkitekturer tyder på att arkitektur inte har en betydande inverkan på modellens prestanda, förutsagt att arkitekturen är djup nog. Modellen misslyckas att klassificera alla spektra och har en tendens för att föredra vissa kategorier över andra. Vi ger förslag för hur de processer vi använt kan förbättras i hopp om att kunna hitta en bättre modell i framtida studier. Dessa förslag är alternativ som inte tagits under analysens utförning. Exempel på dessa är att använda alternativa metoder för att avlägsna grundlinjen samt förbättring av den analytiska metoden för att identifiera icke-tumör spektra genom att avsee klustrens avstånd från varandra.
\title{Foreword}

It is quite concerning that we still struggle with disease despite our growing understanding of the human body and mind. Slowly, these struggles ease as new methods are introduced and we find ourselves in a position to take one small step towards further understanding. Many of these steps may seem insignificant, but they inevitably bring with them the possibility of breakthrough, which in turn has the potential to help people. Artificial intelligence has the potential to aid us in further understanding these diseases by analyzing non-trivial patterns in data. While this thesis fails to provide the field with any substantial understanding of glioma. I will continue to improve our models in hope of improving

This thesis would not have been possible without the supervision and support of Luigia Petre, whose patient attitude and rigorous feedback has provided me great inspiration in writing and expanding each chapter. I would also like to thank Ion Petre, who generously allowed me to take part in the project this thesis covers. I thank my parents and extended family, who all have suffered through lengthy discussions and explanations about this project and other jargon related to my area of study. And my fellow peers, whom I have had the exceptional and delightful privilege of meeting. Our long discussions about our studies have fueled my enthusiasm for years and will continue to do so for years to come. I would especially like to thank Patric Gustafsson, whose hard work and dedication is inspiring. His brilliant thesis motivated me to start this one, and and his support helped me power through to the end. This thesis, not to mention my degree, would not have been possible without you.
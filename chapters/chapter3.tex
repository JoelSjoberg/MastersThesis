In this section the data is presented in greater detail. Due to the low number of samples available, it is necessary to examine each sample in detail to determine it's predictability. The data must be satisfyingly diverse between the given classes and similar within those classes for the predictive model to work appropriately. Should this not be the case, the model will struggle to reach desired performance by either failing to capture basic features of the data or by overfiting to it.


\subsection{Data Representation}
The data of the glioblastoma patients is provided by \textbf{PROVIDERS HERE} and consists of the Raman-spectrum of 45 tumors from separate patients. Multiple samples of tissue was extracted from the same tumor in some cases, yielding several samples for the respective tumors. The samples are sorted by the patient to whom they belong to maintain separation between the patient samples. Moreover, there is large variation with regard to the sample shape within the data. Each sample is a 3-dimensional array of size $(w, h, 1738)$ where $w$ and $h$ are the respective width and height of the sample. This formalization is necessary, as width and height have non-zero variance among different samples. The number 1738 is constant through all samples and represent the length of the Raman-spectrum. Each element inside these arrays is a real number without clear bounds. The largest absolute element found within is $79427.0625$. The maximums of each  
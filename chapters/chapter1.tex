Glioma is a type of brain cancer which manifests within the brain and disrupts brain-functions. Symptoms of such tumors may include headaches, nausea \textbf{Include more!}. The survivability of the cancer is extremely poor with a life expectancy of a few months depending on the patients health and the tumor grade. The world health organization(WHO) cathegorize glioma tumors with grades based on their severity. At present there are four grade types. However a recent study suggests these additional subdivisions of these classes, the introduction of which show promise in expanding knowledge about glioma tumors and aid in treatment selection \cite{cellsubsets}. Such evaluation require in dept knowledge about the tumor tissue and  further examination which may last for weeks after extraction. In this project we aim to optimize this process by introducing a deep learning model capable of producing tumor-type prediction in a matter of minutes. Tissue from tumors of 53 patients is extracted and scanned using Raman spectroscopy to store the Raman spectra. This spectra is the data available for the project and is subject to preprocessing before the model-training.

For this thesis we will consider a dataset consisting Raman Spectrum gathered from 24 different tumor samples, each belonging to a unique patient. Several samples were collected from a select few of the patients with a variable size.

To determine the validity of this experiment, k-means clustering is performed on each sample pair for 
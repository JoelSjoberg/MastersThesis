Glioma is a type of brain cancer which manifests within the brains glial-cells and disrupts brain-functions. The survivability of the cancer is extremely poor with a life expectancy of a few months without treatment to a few years depending on the patients health, the tumor type and grade; rarely do patients survive for five years\cite{gallego2015nonsurgical}\cite{bleeker2012recent}. Gliomas are categorized depending on their glial-cell of origin. Oligendroglioma originates from oligodendrocytes, astocytoma from astocytes and ependymomas originate from ependymal cells. Furthermore, the aforementioned astocytoma-types may develop into glioblastoma multiforme (GBM), the most aggressive form of brain cancer. However, It is also possible for GBM to develop from other brain cells. This cancer is particularly agressive due to quick reappearance in the brain only a short period after surgery\cite{gallego2015nonsurgical}. The heterogeneity of GBM-cells further complicates the healing process by avoiding certain targeted treatments\cite{dirkse2019stem}.

At present there are four grade-types independent from the tumor type defined by the World Health Organization(WHO) used to describe their aggressiveness and growth. These are low-grade (grade I and II) and high-grade(grade III and IV) where glioblastoma is cathegorized by grade IV \cite{bleeker2012recent}\cite{gradesandpriorsubdivision}; these must be examined to determine an appropriate prognosis and line of treatment. However a study by Vigneswaran et al. suggested these grades could be divided further to better describe the features of the tumor e.g. GBM could be divided further to express versions with poor prognosis. Ceccarelli et al.\cite{cellsubsets} introduce alternative subdivisions of these classes which show promise in expanding knowledge about glioma tumors and aid in treatment selection. Such evaluation require in depth knowledge about the tumor tissue and  further examination which may last for weeks after extraction. Ceccarelli et al. define the subdivisions by 6 distinct classes labeled LGm1-6. Their analysis showed IDH-mutations in LGm1-3; furthermore LGm4-6 were IDH-wild-type where a majority of tumors could be labeled as glioblastoma. These clusters are reinforces by the results produced by Vigneswaran et al. The process of determining prognosis and line of treatment have great promise in improving patient outcome by classifying tumors by these subdivisions.

This thesis explains a project whose purpose is to optimize the categorization process by introducing a deep learning model capable of producing tumor-type prediction in a matter of minutes. Tissue from tumors of 53 patients is extracted and scanned using Raman spectroscopy. Raman spectroscopy was invented by Chandrasekhara Venkata Raman and measures the vibrations of molecules by spectral analysis. This method can be executed fairly quickly and can provide chemical information from the spectral light. A laser emits a ray unto the tumor tissue, causing the energy level of the molecules within to change which in turn change their vibration. This vibration is gathered by the instrument and may then be used as data to determine properties of the material\cite{long1977raman}\cite{graves1989practical}. This spectra is the data which the model will use as training and testing data. The choice to utilize Raman spectra in this way is due to the methods success in previous studies where is applied with machine learning algorithms\cite{ramanDL}\cite{ho2019rapid}. The use of Raman spectra is further motivated by Liu et al.\cite{liu2017deep} whose work show promise for deep learning models trained on raw Raman spectra. The advantage the method possess in context of multilabel classification compared to other machine learning methods is shown to be considerable in contrast to Support vector machines, Random forest and K nearest neighbors\cite{liu2017deep}. In chapter two the preliminary theoretical background for machine learning is covered along with the necessary mathematical definitions by which these methods are defined. Understanding the underlying definitions is necessary to validate and confirm the results with expected and public results in the field. Chapter three explains the exploration methods in detail to give further understanding of the data on which this project is based. The chapter begins by introducing the concrete shape of the data and proceeds to display the methods in the order they were utilized, the use of unsupervised learning and feature selection is motivated. The chapter concludes by presenting the results of the exploration-stage. Chapter four presents the deep learning model alongside the performance it yields and the expected impact it will have on the medical field. The primary focus of this thesis will be placed on chapter three and four as they cover the purpose of the project. Chapter five concludes this thesis by giving suggestions to further study along with arguments for and against the use of machine learning in the medical field.


The mammalian brain consists primarily of glial cells, previously thought to be insignificant in terms of the brains computational functionality, only lending structural support to the neurons. Recent studies have disputed this and suggests the cells importance to the nervous system is greater than once thought, though their actual function is still a matter of speculation \ref{glialcells}. Glioma is a type of brain cancer which manifests within the brains glial-cells and disrupts brain-functions. The survivability of the cancer is extremely poor with a life expectancy of a few months without treatment to a few years depending on the patients health, the tumor type and grade; rarely do patients survive for five years \cite{gallego2015nonsurgical}\cite{bleeker2012recent}. Gliomas are categorized depending on their glial-cell of origin. There are four types of glial cells (also called neurolia or simply glia) Oligodendrocytes, astocytes, ependymal cells and microglia. Oligendroglioma originates from oligodendrocytes, astocytoma from astocytes and ependymomas originate from ependymal cells. Furthermore, the aforementioned astocytoma-types may develop into glioblastoma multiforme (GBM), the most aggressive form of brain cancer which is may communicate with  microglia to increase tumor growth \ref{maas2020glioblastoma}. However, It is also possible for GBM to develop from other brain cells. This cancer is particularly aggressive due to its quick reappearance in the brain only a short period after surgery \cite{gallego2015nonsurgical}. The heterogeneity of GBM-cells further complicates the healing process, due to poor response to targeted treatments \cite{dirkse2019stem}.

At present, the World Health Organization (WHO) has defined four levels of cancer severity used to describe their aggressiveness and tumor growth (these levels are also called grades). These are low-grade (grade I and II) and high-grade (grade III and IV) glioblastoma is cathegorized by grade IV \cite{bleeker2012recent, gradesandpriorsubdivision}; and these must be examined to determine an appropriate prognosis and line of treatment. However a study by Vigneswaran et al. \ref{gradesandpriorsubdivision} suggested these grades could be divided further to better describe the features of the tumor and express versions with poor prognosis. This suggestion is also supported by Hirose et al. \ref{hirose2013subgrouping}. Ceccarelli et al.\cite{cellsubsets} introduce alternative subdivisions of these classes which show promise in expanding knowledge about glioma tumors and aid in treatment selection. Such evaluation require in depth knowledge about the tumor tissue and  further examination which may last for weeks after extraction. Ceccarelli et al. define the subdivisions by 6 distinct classes, labeled LGm1-6. Their analysis showed IDH-mutations in LGm1-3; furthermore LGm4-6 were IDH-wild-type where a majority of tumors could be labeled as glioblastoma. These clusters are reinforced by the results produced by Vigneswaran et al. The process of determining a prognosis and a line of treatment has great promise in improving patient outcome by classifying tumors into these subdivisions.

This thesis is the result of a project whose purpose is to optimize the categorization process based on a deep learning model capable of producing tumor-type prediction in a matter of minutes. The project relies on tissue from tumors extracted from 53 patients and scanned using Raman spectroscopy. Raman spectroscopy was invented by Chandrasekhara Venkata Raman and measures the vibrations of molecules by spectral analysis. This method can be executed fairly quickly and can provide chemical information from the spectral light. A laser emits a ray unto the tumor tissue, causing the energy level of the molecules within to change, which in turn changes their vibration. This vibration is gathered by the instrument and provides information regarding the molecular properties of the material \cite{long1977raman}\cite{graves1989practical}. This spectra is the data which the model uses as training and testing data. The choice of using Raman spectra in this way is due to the method's success in previous studies of Raman spectra using machine learning \cite{ramanDL}\cite{ho2019rapid}. The use of Raman spectra is further motivated by Liu et al. \cite{liu2017deep}, whose work show promise for deep learning models trained on raw Raman spectra. The advantage of this method in context of multilabel classification is considerable, when compared to other machine learning methods such as Support Vector Machines, Random Forest and K-nearest neighbor \cite{liu2017deep}.

We proceed as follows, Chapter 2 presents the preliminary theoretical background for machine learning, along with the necessary mathematical definitions by which these methods are defined. Understanding the underlying definitions is necessary to validate and confirm the results. In Chapter 3, we discuss the exploration methods in detail, to give further understanding of the data on which this project is based. The chapter begins by introducing the concrete shape of the data and proceeds to display the methods in the order they were utilized. The use of unsupervised learning and feature selection are also explained and motivated. In Chapter 4, we present the deep learning model, the performance it yields. In Chapter 5 concludes this thesis by giving suggestions to further study along with arguments for and against the use of machine learning in the medical field. The thesis is concluded in Chapter 6 with a discussion of the impact the deep learning model will have on the medical field. The primary focus of this thesis will be placed on Chapter 3 and 4 as they cover the purpose of the project.
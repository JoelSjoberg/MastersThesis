Glioma is a type of brain cancer which manifests within the brains glial-cells and disrupts brain-functions. Symptoms of such tumors may include headaches, nausea, loss of memory, seizures, decline in brain function among others defined by the Mayo Clinic\cite{mayoglioma}. The survivability of the cancer is extremely poor with a life expectancy of a few months depending on the patients health and the tumor type and grade. The primary types gliomas are cathergorized in depend on their glialcell of origin and as of yet undefined genetic information, though studies have found connections with some genes e.g. IDH1 and IDH2\cite{cellsubsets}. Oligendroglioma originates from oligodendrocytes, Astocytoma from astocytes and Ependymomas originate from ependymal cells\cite{nihglioma}. At present there are four grade types independent from the tumor type used to describe their aggressiveness and growth. These are low-grade (grade I and II) and high-grade(grade III and IV) and must be examined to determine an appropriate line of treatment\cite{hopkinsglioma}. However a recent study by Ceccarelli et al. suggests an alternative subdivisions of these classes, the introduction of which show promise in expanding knowledge about glioma tumors and aid in treatment selection\cite{cellsubsets}. Such evaluation require in depth knowledge about the tumor tissue and  further examination which may last for weeks after extraction. In this project we aim to optimize this process by introducing a deep learning model capable of producing tumor-type prediction in a matter of minutes. Tissue from tumors of 53 patients is extracted and scanned using Raman spectroscopy. This spectra is the data available for the project and is subject to preprocessing before the model-training. Raman spectra is stored from the scattered light reflected from a material under the emitting laser. This method is used to extract chemical information of the tumor tissue and has been successful in previous studies where machine learning has been applied\cite{ramanUseful0}.
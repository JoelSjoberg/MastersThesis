The mammalian brain contains so-called neurons and glial cells. Historically, it was believed that the brain contained ten times as many glial cells as neurons, but recent studies suggest the number of neurons is equal to the number of glial cells \cite{von2016search}. Glial cells were also previously thought to be insignificant in terms of the brains computational functionality, only lending structural support to the neurons. Recent studies have disputed this and suggests their contribution to the nervous system is greater than once thought, though their actual function is still a research question. Glioma is a type of brain cancer which manifests within the glial cells and disrupts brain functions. The survivability of the cancer is extremely poor, with a life expectancy of a few months (without treatment) to a few years depending on the patients health, the tumor type and cancer severity; rarely do patients survive for longer than five years \cite{glialcells, gallego2015nonsurgical, bleeker2012recent}. Gliomas are categorized depending on their glial-cell of origin. There are four main types of glial cells (also called neuroglia or simply glia): oligodendrocytes, astocytes, ependymal cells and microglia. Oligendroglioma originates from oligodendrocytes, astocytoma from astocytes and ependymomas originate from ependymal cells. Furthermore, astocytoma-types may develop into glioblastoma multiforme (GBM), the most aggressive form of brain cancer; this may even communicate with  microglia to increase tumor growth \cite{maas2020glioblastoma}. It is also possible for GBM to develop from other brain cells \cite{glialcells}. This cancer is particularly aggressive, due to its quick reappearance in the brain, only a short period after surgery \cite{gallego2015nonsurgical}. The heterogeneity of GBM-cells further complicates the healing process, due to poor response to targeted treatments \cite{dirkse2019stem}.

The World Health Organization (WHO) has defined four levels (or "grades") of cancer severity used to describe the cancer aggressiveness and tumor growth. Grades I and II are considered low-grade and grades III and IV are considered high-grade. Glioblastoma is categorized as a grade IV cancer \cite{bleeker2012recent, gradesandpriorsubdivision}; these grades are used to determine an appropriate prognosis and line of treatment. A study by Vigneswaran et al. \cite{gradesandpriorsubdivision} suggests these grades could be divided further to better describe the features of the tumors. This suggestion is also supported by Hirose et al. \cite{hirose2013subgrouping}. Ceccarelli et al. \cite{cellsubsets} introduce alternative subdivisions of these classes, which show promise in expanding knowledge about glioma tumors and aid in treatment selection. Such evaluations require in-depth knowledge about the tumor tissue in addition to further examination of it, which may last for weeks after the tumor extraction. Ceccarelli et al. define the subdivisions by six distinct classes, labeled LGm1-6. Their analysis showed IDH mutations in LGm1-3; as the name suggests, IDH mutations refer to mutations in the IDH1 or IDH2 genes. These mutations are shown to be significant in a variety of cancers, including glioma \cite{dang2016idh}. % Furthermore, LGm4-6 were IDH wild-type, where a majority of tumors could be labeled as glioblastoma. IDH wild-type refers to IDH genes with no mutations, but they are often correlated with poor prognosis in high-grade glioma. 
These subdivisions are reinforced by the results produced by Vigneswaran et al. The process of determining a prognosis and a line of treatment using the subdivisions, shows great promise in improving patient outcome.

This thesis is the result of a project whose purpose is to optimize the categorization process, based on a deep learning model capable of predicting tumor-types in a matter of minutes. The project relies on tissue from tumors extracted from 45 patients and scanned using Raman spectroscopy. Raman spectroscopy was introduced by Chandrasekhara Venkata Raman and measures the vibrations of molecules by spectral analysis. This method can be executed fairly quickly and can provide chemical information from the spectral light. A laser emits a ray unto the tumor tissue, causing the energy level of the molecules within to change, which in turn changes their vibrations. This vibrations are gathered by the instrument and provide information regarding the molecular properties of the material \cite{long1977raman, graves1989practical}. This spectra is the data which the model uses as training and testing data. The choice of using Raman spectra in this way is due to the method's success in previous studies of Raman spectra using machine learning \cite{ramanDL, ho2019rapid}. The use of Raman spectra is further motivated by Liu et al. \cite{liu2017deep}, whose work show promise for deep learning models trained on raw Raman spectra. The advantage of this method in the context of multilabel classification seem considerable, when compared to other machine learning methods such as Support Vector Machines, Random Forest and K-nearest neighbor \cite{liu2017deep}.

This thesis aims to analyze the spectra extracted from all patient samples in an attempt to automate outlier detection.% The samples are examined by statistical methods designed for outlier detection. Hierarchical clustering and K-means clustering are applied to the samples to divide the spectra into subsets which we find identifies many outliers. 
We examine the samples by applying statistical methods, hierarchical clustering and K-means clustering; this produces subdivisions of spectra and identifies outliers. These results are compared to the results of a criterion for finding outliers in the data (defined by the data provider). The method most suitable for this purpose is then used to remove the outliers. Following the removal of the outliers, we present a pre-processing pipeline which will be used to prepare the data for machine learning applications such as Artificial Neural Networks or Random Forests. The features which best divide the data into the six LGm classes are extracted. These features drastically reduce the size of the spectra which are analyzed for prognosis, which in turn reduces the examination time. We thus aim to provide a clear way of preparing Raman spectra for machine learning applications and provide the most important features those spectra consist of. Suggestions and a discussion for how these methods may be improved, which alternative methods could be tested instead and eventual limits to this project are also given for future consideration.

The thesis is structured as follows. Chapter 2 presents the preliminary background for the statistical methods used in the project, along with the necessary mathematical definitions by which these methods are defined. Among these, we discuss the notions of mean, standard deviation and analysis of variance (ANOVA). Understanding the underlying definitions and consequences is necessary to validate and confirm the results. Therefore, the chapter also presents the definition of supervised and unsupervised learning. The formal definitions of K-means clustering and hierarchical clustering are presented. In Chapter 3, we discuss the analysis methods in detail, to give further understanding of the data on which this project is based. The chapter begins by introducing the concrete shape of the data. Feature selection is applied to the data and the results are examined. The majority of this chapter is based on the visual analysis of the outlier detection. This is done by applying the statistical methods and the clustering methods to the samples. The results of each method are analyzed in comparison to the criterion defined by the data provider in greater detail to form an argument for or against the method in question. The chapter ends by removing the outliers using the optimal method and performing feature selection once more on the data devoid of outliers. In Chapter 4, we present our suggestion as well as arguments for the pre-processing pipeline to prepare the data for machine learning. A Neural Network is created and trained on the curated data. The performance of the architecture is measured and presented. The thesis is concluded in Chapter 5, where we discuss improvements and suggestions to out methods. We also provide suggestions for alternative methods for feature selection and pre-processing for future study and tests.
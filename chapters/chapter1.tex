Glioma is a type of brain cancer which manifests within the brains glial-cells and disrupts brain-functions. Symptoms of such tumors may include headaches, nausea, loss of memory, seizures, decline in brain function among others defined by the Mayo Clinic\cite{mayoglioma}. The survivability of the cancer is extremely poor with a life expectancy of a few months depending on the patients health, the tumor type and grade; rarely do patients survive longer than a few years\cite{gallego2015nonsurgical}\cite{bleeker2012recent}. Gliomas are cathergorized depending on their glial-cell of origin. Oligendroglioma originates from oligodendrocytes, astocytoma from astocytes and ependymomas originate from ependymal cells\cite{nihglioma}. Furthermore, the aforementioned astocytoma-types may develop into glioblastoma multiforme (GBM), the most aggressive form of brain cancer. However, It is possible for GBM to develop from other brain cells.\cite{gallego2015nonsurgical}.

At present there are four grade-types independent from the tumor type used to describe their aggressiveness and growth. These are low-grade (grade I and II) and high-grade(grade III and IV) where glioblastoma is cathegorized by grade IV \cite{bleeker2012recent}; these must be examined to determine an appropriate prognosis and line of treatment \cite{hopkinsglioma}. However a recent study by Ceccarelli et al.\cite{cellsubsets} introduce alternative subdivisions of these classes which show promise in expanding knowledge about glioma tumors and aid in treatment selection. Such evaluation require in depth knowledge about the tumor tissue and  further examination which may last for weeks after extraction. Ceccarelli et al. define the subdivisions by 6 distinct classes labeled LGm1-6. Their analysis showed IDH-mutations in LGm1-3; furthermore LGm4-6 were IDH-wild-type where a majority of tumors could be labeled as glioblastoma.

In this project we aim to optimize this process by introducing a deep learning model capable of producing tumor-type prediction in a matter of minutes. Tissue from tumors of 53 patients is extracted and scanned using Raman spectroscopy. This spectra is the data which the model will use as training and testing data respectively. Raman spectra is stored from the scattered light reflected from a material under the emitting laser. This method is used to extract chemical information of the tumor tissue and has been successful in previous studies where machine learning has been applied\cite{ramanDL}\cite{ho2019rapid}. The use of raman spectra is further motivated by Liu et al.\cite{liu2017deep} whose work show promise for deep learning models trained on raw raman spectra and the advantage the method possess in context of multilable classification compared to other machine learning methods e.g. Support vector machines, Random forest and K nearest neighbors.
\documentclass[a4paper, 12pt, oneside]{book}
\usepackage[english]{babel} % choose the language
\usepackage[utf8]{inputenc} % åäö
\usepackage[T1]{fontenc}
\usepackage[a4paper, inner=3cm, outer=3cm, top=3cm, bottom=3cm]{geometry}
\usepackage[onehalfspacing]{setspace} 
\usepackage{float}
\usepackage{mathptmx} % Times New Roman
\usepackage{tikz} % Support for pictures in latex format -> convert svg files into tikz format at first
\usepackage{amsfonts}
\usepackage[toc,page]{appendix}
\usepackage{etoolbox} % this one makes a workaround with page numbering possible
\usepackage{subfig}
\usepackage{csquotes} % BibLaTex stuff
\usepackage[toc, page]{appendix}
\usepackage{geometry}
 \geometry{
 a4paper,
 total={170mm,257mm},
 left=40mm,
 top=25mm,
 }

% Table of contents
\setcounter{secnumdepth}{5}
\setcounter{tocdepth}{5}

\begin{document}

\pagestyle{empty}    
\begingroup
\patchcmd{\chapter}{plain}{empty}{}{}

%------ Cover page stuff 

\begin{titlepage}
\vspace*{144pt}
\begin{center}
\Huge\bf The analysis and preprocessing of raman spectra of glioma cells %------ NAME OF THE THESIS HERE!!!


\end{center}
\enlargethispage{3cm}
\vfill

\hfill
\begin{tabular}[t]{l@{}}%{\raggedleft%
\textit{Author:} Joel Sjöberg 38686\\ % YOUR NAME AND STUDENT ID!
Masters thesis in Computer Science\\ % BAC/MASTER THESIS AND MAIN SUBJECT
\textit{Supervisor:} Luigia Petre\\ % NAME OF YOUR SUPERVISOR
The Faculty Of Science And Engineering\\ % FACULTY
Åbo Akademi University\\ 
2021\\ % YEAR
\\
\\
\\
\\
\end{tabular}
%}%
\end{titlepage}

%------ end of cover page stuff

\tableofcontents 


\endgroup % end of TeX group

\clearpage
\pagestyle{plain}      
\pagenumbering{arabic} % Page numbering starts from here
% Example of structure
\chapter*{Foreword}
\title{Foreword}

It is quite concerning that we still struggle with disease despite our growing understanding of the human body and mind. Slowly, these struggles ease as new methods are introduced and we find ourselves in a position to take one small step towards further understanding. Many of these steps may seem insignificant, but they inevitably bring with them the possibility of breakthrough, which in turn has the potential to help people. Artificial intelligence has the potential to aid us in further understanding these diseases by analyzing non-trivial patterns in data. While this thesis fails to provide the field with any substantial understanding of glioma. I will continue to improve our models in hope of improving

This thesis would not have been possible without the supervision and support of Luigia Petre, whose patient attitude and rigorous feedback has provided me great inspiration in writing and expanding each chapter. I would also like to thank Ion Petre, who generously allowed me to take part in the project this thesis covers. I thank my parents and extended family, who all have suffered through lengthy discussions and explanations about this project and other jargon related to my area of study. And my fellow peers, whom I have had the exceptional and delightful privilege of meeting. Our long discussions about our studies have fueled my enthusiasm for years and will continue to do so for years to come. I would especially like to thank Patric Gustafsson, whose hard work and dedication is inspiring. His brilliant thesis motivated me to start this one, and and his support helped me power through to the end. This thesis, not to mention my degree, would not have been possible without you.

\chapter*{Abstract}
\title{Abstract}

But not abstract enough.

%\chapter{Introduction}
%
\title{Motivation}

This might be removed entirely, all I can come up with is already explained in the project description. Maybe project description can just be introduction... That would probably look nicer, since it presents the project on a high level, the actual description is the thesis itself after all.


\bibstyle{biblatex}
\bibdata{referenser,bib}
\citation{biblatex-control}

\chapter{Introduction}
The mammalian brain contains so-called neurons and glial cells. Historically, it was believed that the brain contained ten times as many glial cells as neurons, but recent studies suggest the number of neurons is equal to the number of glial cells \cite{von2016search}. Glial cells were also previously thought to be insignificant in terms of the brains computational functionality, only lending structural support to the neurons. Recent studies have disputed this and suggests their contribution to the nervous system is greater than once thought, though their actual function is still a research question. Glioma is a type of brain cancer which manifests within the glial cells and disrupts brain functions. The survivability of the cancer is extremely poor, with a life expectancy of a few months (without treatment) to a few years depending on the patients health, the tumor type and cancer severity; rarely do patients survive for longer than five years \cite{glialcells, gallego2015nonsurgical, bleeker2012recent}. Gliomas are categorized depending on their glial-cell of origin. There are four main types of glial cells (also called neuroglia or simply glia): oligodendrocytes, astocytes, ependymal cells and microglia. Oligendroglioma originates from oligodendrocytes, astocytoma from astocytes and ependymomas originate from ependymal cells. Furthermore, astocytoma-types may develop into glioblastoma multiforme (GBM), the most aggressive form of brain cancer; this may even communicate with  microglia to increase tumor growth \cite{maas2020glioblastoma}. It is also possible for GBM to develop from other brain cells \cite{glialcells}. This cancer is particularly aggressive, due to its quick reappearance in the brain, only a short period after surgery \cite{gallego2015nonsurgical}. The heterogeneity of GBM-cells further complicates the healing process, due to poor response to targeted treatments \cite{dirkse2019stem}.

The World Health Organization (WHO) has defined four levels (or "grades") of cancer severity used to describe the cancer aggressiveness and tumor growth. Grades I and II are considered low-grade and grades III and IV are considered high-grade. Glioblastoma is categorized as a grade IV cancer \cite{bleeker2012recent, gradesandpriorsubdivision}; these grades are used to determine an appropriate prognosis and line of treatment. A study by Vigneswaran et al. \cite{gradesandpriorsubdivision} suggests these grades could be divided further to better describe the features of the tumors. This suggestion is also supported by Hirose et al. \cite{hirose2013subgrouping}. Ceccarelli et al. \cite{cellsubsets} introduce alternative subdivisions of these classes, which show promise in expanding knowledge about glioma tumors and aid in treatment selection. Such evaluations require in-depth knowledge about the tumor tissue in addition to further examination of it, which may last for weeks after the tumor extraction. Ceccarelli et al. define the subdivisions by six distinct classes, labeled LGm1-6. Their analysis showed IDH mutations in LGm1-3; as the name suggests, IDH mutations refer to mutations in the IDH1 or IDH2 genes. These mutations are shown to be significant in a variety of cancers, including glioma \cite{dang2016idh}. % Furthermore, LGm4-6 were IDH wild-type, where a majority of tumors could be labeled as glioblastoma. IDH wild-type refers to IDH genes with no mutations, but they are often correlated with poor prognosis in high-grade glioma. 
These subdivisions are reinforced by the results produced by Vigneswaran et al. The process of determining a prognosis and a line of treatment using the subdivisions, shows great promise in improving patient outcome.

This thesis is the result of a project whose purpose is to optimize the categorization process, based on a deep learning model capable of predicting tumor-types in a matter of minutes. The project relies on tissue from tumors extracted from 45 patients and scanned using Raman spectroscopy. Raman spectroscopy was introduced by Chandrasekhara Venkata Raman and measures the vibrations of molecules by spectral analysis. This method can be executed fairly quickly and can provide chemical information from the spectral light. A laser emits a ray unto the tumor tissue, causing the energy level of the molecules within to change, which in turn changes their vibrations. This vibrations are gathered by the instrument and provide information regarding the molecular properties of the material \cite{long1977raman, graves1989practical}. This spectra is the data which the model uses as training and testing data. The choice of using Raman spectra in this way is due to the method's success in previous studies of Raman spectra using machine learning \cite{ramanDL, ho2019rapid}. The use of Raman spectra is further motivated by Liu et al. \cite{liu2017deep}, whose work show promise for deep learning models trained on raw Raman spectra. The advantage of this method in the context of multilabel classification seem considerable, when compared to other machine learning methods such as Support Vector Machines, Random Forest and K-nearest neighbor \cite{liu2017deep}.

This thesis aims to analyze the spectra extracted from all patient samples in an attempt to automate outlier detection.% The samples are examined by statistical methods designed for outlier detection. Hierarchical clustering and K-means clustering are applied to the samples to divide the spectra into subsets which we find identifies many outliers. 
We examine the samples by applying statistical methods, hierarchical clustering and K-means clustering; this produces subdivisions of spectra and identifies outliers. These results are compared to the results of a criterion for finding outliers in the data (defined by the data provider). The method most suitable for this purpose is then used to remove the outliers. Following the removal of the outliers, we present a pre-processing pipeline which will be used to prepare the data for machine learning applications such as Artificial Neural Networks or Random Forests. The features which best divide the data into the six LGm classes are extracted. These features drastically reduce the size of the spectra which are analyzed for prognosis, which in turn reduces the examination time. We thus aim to provide a clear way of preparing Raman spectra for machine learning applications and provide the most important features those spectra consist of. Suggestions and a discussion for how these methods may be improved, which alternative methods could be tested instead and eventual limits to this project are also given for future consideration.

The thesis is structured as follows. Chapter 2 presents the preliminary background for the statistical methods used in the project, along with the necessary mathematical definitions by which these methods are defined. Among these, we discuss the notions of mean, standard deviation and analysis of variance (ANOVA). Understanding the underlying definitions and consequences is necessary to validate and confirm the results. Therefore, the chapter also presents the definition of supervised and unsupervised learning. The formal definitions of K-means clustering and hierarchical clustering are presented. In Chapter 3, we discuss the analysis methods in detail, to give further understanding of the data on which this project is based. The chapter begins by introducing the concrete shape of the data. Feature selection is applied to the data and the results are examined. The majority of this chapter is based on the visual analysis of the outlier detection. This is done by applying the statistical methods and the clustering methods to the samples. The results of each method are analyzed in comparison to the criterion defined by the data provider in greater detail to form an argument for or against the method in question. The chapter ends by removing the outliers using the optimal method and performing feature selection once more on the data devoid of outliers. In Chapter 4, we present our suggestion as well as arguments for the pre-processing pipeline to prepare the data for machine learning. A Neural Network is created and trained on the curated data. The performance of the architecture is measured and presented. The thesis is concluded in Chapter 5, where we discuss improvements and suggestions to out methods. We also provide suggestions for alternative methods for feature selection and pre-processing for future study and tests. % chapters actually begins from 2, motivation is chapter 1

\chapter{Theoretical Background}
In this chapter we review the concepts on which this thesis is based. We first cover the statistical concepts and unsupervised methods necessary for understanding the analysis performed in Chapter 3. We then proceed by reviewing common concepts and specific methods within machine learning which is the central theme in Chapter 4.% We then proceed by introducing statistical concepts required for understanding the methods employed in the project. We then review common concepts and specific methods within machine learning which is the central theme in this thesis.


\section{Statistics}

In this section we discuss the basic statistical concepts required for understanding the methods used in this thesis. We first explain the mean and the standard deviation. The mean is a value used to describe the average value of a population. A population is the term used to describe the complete collection of elements in some context e.g. all people alive, all numbers in an interval etc. The mean is an important concept in statistics as it is often used as a characteristic of the elements found in the population under analysis when its elements are quantitative \cite{rowntree1981statistics}. The mean is calculated by the sum of each element divided by the number of elements in said population. The mean $\mu$ of a population of $n$ numbers $L = \{L_1, ..., L_n\}$, is calculated as expressed in formula (\ref{eqn:mean}).

\begin{equation}
\label{eqn:mean}
\mu = \sum_{i=1}^n \frac{L_i}{n}
\end{equation}

The mean of a population is often used in association with the standard deviation. The standard deviation is the square root of the variance of that population. The variance is an expression for the scaled summed squares of differences from the mean of the entire population. Calculating the square root then produces a value expressing the dispersion of elements within the population around the mean. The variance is calculated by measuring the summed squared distance between each element within the population and the mean, divided by the number of elements in the population to scale the sum. In the case where the entire population is impossible to analyze or unknown, samples are extracted from the population (i.e. subgroups of elements randomly taken from the population). The sample variance changes slightly form the population variance; it divides the summed squares of differences by the number of elements in the sample subtracted by one. Subtracting the original denominator by one is called "Bessel's correction". The correction is based on the fact that, if the population mean is unknown (as is often the case when samples are gathered), then the mean used in calculating the variance will only be an estimation. Subtraction by one is performed to avoid bias towards the sample mean and get an unbiased estimation of the population variance \cite{so2008sample, nobach2020practical}. The standard deviation $\sigma$ of a sample is thus calculated as expressed in formula (\ref{eqn:std}).

\begin{equation}
\label{eqn:std}
 \sigma = \sqrt{\frac{\sum_{i = 1}^{n} (\mu - L_i)^2}{n-1}}
\end{equation}


\subsection{The standard deviation test}
The mean and standard deviation can be used to detect outliers in a sample of data points drawn from an unknown population. The Gaussian distribution (also called the normal distribution) is used in association with the mean and the standard deviation. If the frequency which elements appear within a sample are more likely to be close to the mean than far away from it, we say that the elements within that sample are normally distributed. If an element differs from the mean by more than three standard deviations, the possibility of that element not belonging to that distribution is extremely high. Such elements, which likely do not belong to the population, are called outliers. The method which detects outliers by measuring the distance between each element and the mean is referred to, in this thesis, as the standard deviation test (SDT). The test assumes that elements within the collection are normally distributed with some mean $\mu_c$ and some standard deviation $\sigma_c$ (where c is the collection from which they are measured). By computing the z-score of the elements, the data is transformed into a standardized form through standardization. Z-score standardization subtracts the mean from all elements in the collection and divides the difference by the standard deviation, as expressed in equation (\ref{eqn:zscore}) \cite{geary1935ratio}.

\begin{equation}
\label{eqn:zscore}
 \frac{L_i - \mu_c} {\sigma_c}
\end{equation}

Following standardization, the entire sample will have a mean of zero and a standard deviation of one. Each element in the sample can then be measured using the standard deviation (one) as unit; outliers can then be discarded from the sample by removing elements which have an absolute value of 3 or higher.


\subsection{The interquartile range method}
An alternative method suited for outlier detection is the interquartile range method (IRM). IRM is based on analyzing the sample by its median, which is the center element of the sorted sample. It is applied by sorting the sample elements in ascending order and organizing the sample into four percentiles, each percentile containing $25\%$ of the sample. These percentiles are called the $q_{25}$, $q_{50}$, $q_{75}$, $q_{100}$ percentiles. Two consecutive quartiles center around $50\%$ of the sample contents, the $50\%$ in the center is referred to as the interquartile range. IRM requires two values from the sample: the highest value of the $25$th percentile and the highest value of the $75$th percentile. The former value is gained by taking the biggest value of the $q_{25}$ first elements from the sorted collection, where $q_{25}$ is the first $25\%$ of the number of elements in $L$ (calculated by $0.25 \cdot n$). The latter value is gained in a similar fashion, with the exception that the highest value is taken from the first $75\%$ of the sorted data. Two \textit{cut-off} points are then defined, by multiplying each of the two values by a constant $k$ (which is called the \textit{cut-off} constant). Elements can then be labeled as outliers if those elements fall below the lower \textit{cut-off} point or above the higher \textit{cut-off} point. \cite{vinutha2018detection, walfish2006review, dovoedo2015boxplot}.

\subsection{Analysis of variance}

The analysis of variance (ANOVA) is a method for statistical analysis developed by Ronald Fisher to analyze the difference among sample means in a collection of samples. It is based on the null-hypothesis stating that the sample means of two or more samples are the same. The analysis then yields a F-value and a p-value which are used with the F-distribution to accept or reject the null-hypothesis. The analysis assumes the population samples drawn are normally distributed and that the samples are independent of each other. It also assumes the standard deviations and variances are roughly equal among the samples \cite{scheffe1999analysis}.

The analysis is performed by calculating the mean of each sample. The summed square of differences from each mean are then calculated for their respective sample, the result is then subtracted by the squared sum of the elements within the sample divided by the number of elements in said sample. This calculation for sample $S$ is formally expressed in equation (\ref{eqn:SS}), where $\mu_S$ is the mean of sample $S$.

\begin{equation}
\label{eqn:SS}
 \sum_{s \in S} (s - \mu_S)^2 - \frac{(\sum_{s}^{S} s)^2}{|S|} 
\end{equation}

This is performed once for all samples drawn from the population, the results are then summed together into a sum of sample sums (SS). The calculation is also performed once on the total collection of all elements from all samples, the result is stored in the total sum of sample sums (SST). The total sum of squares is the sum of SS and the sum of squared distances from each sample mean to the total mean (SSM). SSM is therefore calculated by subtracting SST and SS. The analysis also requires divisions by two values called the degrees of freedom. They are calculated as follows: $d_1$ is the number of samples subtracted by one and $d_2$ is the number of elements in the total collection subtracted by the number of samples. The F-value is calculated by a fraction of two fractions. Let  $K_1$ be the fraction $\frac{SSM}{d_1}$ and $K_2$ be the fraction $\frac{SS}{d_2}$, the \textit{F-value} is defined as expressed in formula (\ref{eqn:Fval}).

\begin{equation}
\label{eqn:Fval}
 F = \frac{\frac{SSM}{d_1}}{\frac{SS}{d_2}} = \frac{K_1}{K_2}
\end{equation}

The quotient yielded by the fraction in formula (\ref{eqn:Fval}) can then be inserted into a pre-calculated table of the F-distribution using the degrees of freedom to yield the p-value. A small p-value (i.e. lower than $0.05$) indicates that the null-hypothesis may be rejected with relative certainty whereas a high p-value indicates the null-hypothesis holds and should be accepted with relative certainty \cite{lowry2014concepts}.


\section{Machine Learning}

Machine learning is the practice of computing models for relationships between sets of data. The field has garnered significant interest within academia and industry alike due to the promising results in applications for which deterministic algorithms have proven difficult or impossible to make. Examples of such applications are computer vision, natural language processing and personalized advertising, to name a few \cite{sebe2005machine, allen2003natural, malheiros2012too}. There are two main paradigms for learning: Supervised learning (using labeled data to approximate models) and unsupervised learning (finding patterns within the data itself). 

Models are used to great length within many scientific domains. In the context of machine learning, a model can be seen as a data structure made out of constant parameters combined with an algorithm which utilizes the data structure to produce predictions given an input vector (the input can also be in the form of a multi-dimensional matrix). 

The model can be represented mathematically as a collection of structures in the form of vectors or matrices, the elements of which are referred to as parameters. A model can consist of learnable parameters $\theta$ and non-learnable parameters (often generated by stochastic initialization if used). The model computes a function $f$ to yield a prediction $y$ by applying the algorithm to the parameters given an input example $x$ (which can be a vector or a matrix) drawn from the data set $X$. Let the dimensionality of the input $x$ be equal to the dimensionality of $\theta$. An example of a model prediction, where the algorithm produces predictions through addition, is given by formula (\ref{eqn:simplemodel}).

\begin{equation}
\label{eqn:simplemodel}
 y = x_0\theta_0 + x_1\theta_1 + ... + x_n\theta_n
\end{equation}

Machine learning then, is the practice of changing (also known as tuning) $\theta$ by introducing small changes to the elements within $\theta$. This is done to minimize a loss function which computes the error (or loss) given $y$. The process of changing $\theta$ is known as the training process and is central to machine learning. 

In the training process for supervised learning, the data gathered for the model is separated into three sets. These sets are referred to as the training set, validation set and test set. They are randomly collected samples from the common data set such that the intersection between any two sets is empty. The purpose of the training process is to train the model on the training data and use the validation and test sets as a means to validate the model performance on data not encountered during the training process. Supervised learning requires that the examples used have a label which the model tries to predict (data which possess labels are called labeled data). We say that a model generalizes well to the data if the training process allows the model to perform well on unseen data. If the model manages to perform well on the training set but fails to generalize, the model is said to overfit to the data. 

Unsupervised learning is a learning paradigm which does not rely on the use of labeled data. Instead, the paradigm focuses on organizing the data in a way that minimizes the loss function. Predictions can then be performed by evaluating the way the data has been organized by some method related to the problem context. The problem context is usually framed by two problem categories. These categories are regression and classification. Regression is the task of predicting continuous values given either continuous or discrete data i.e. predicting stock prices given information about the current economical situation or predicting the number of patients in a hospital during a pandemic given the density of the population. Classification aims to group different examples into categories (or classes) i.e. predicting whether an image contains a dog or a cat or which category a given tumor belongs to. Both supervised and unsupervised learning are used in this project.


\subsection{K-means Clustering}

Clustering is an unsupervised learning method whose primary use is in grouping data into sets. In this thesis we consider the \textit{K-means} clustering algorithm. The following is a formal definition of \textit{K-means} clustering as defined by MacQueen \cite{macqueen}. Given a set $E_n$ of $n$-dimensional points (where $n \in \mathbb{N}$) and a desired amount of partitions $k$ of $E_n$, partition the elements of $E_n$ into $k$ sets. The partitions are stored in a superset $S$ such that $S = \{S_1, S_2, ... S_k\}$. The partitioning of $E_n$ is performed by randomly initializing $k$ $n$-dimensional points as randomly selected points within $E_n$, these are the initial clusters. We define the set of clusters $V$ with elements $v$, where $v_i$ is the i:th cluster center and $i \in [1, k] \cap \mathbb{N}$. The partitioning of the elements $x \in E_n$ into their respective partition $S_i$ is performed by computing the closest cluster center for all elements in $E_n$. Let $T_i$ where $i \in [1, k] \cap \mathbb{N}$ be the set of elements $x \in E_n$ such that the distance from the element to the relevant cluster center is minimal; $T_i$ is defined by formula (\ref{eqn:Ti}).

\begin{equation}
\label{eqn:Ti}
T_i = \{x : x \in E_n | (|x - v_i| \leq |x - v_j|)\} (j \in [1, k] \cap \mathbb{N})
\end{equation}

For centers that share equal distance to any given $x$, the cluster with the smallest index is chosen as the containing set. This is performed by iteratively defining $S_i$ as the intersection of $T_i$ and the points which are not in any prior partitioned sets i.e. for $S_j$ where $j < i$. This is denoted by the set complement $S_i^c$ for all elements not in $S_i$. Let $S_1$ be defined by $T_1$, then the partitions $S_i \in S$ for $i \in [2, k] \cap \mathbb{N}$ are defined by formula (\ref{eqn:Si}).

\begin{equation}
\label{eqn:Si}
S_i = T_i \cap \bigcap_{j=1}^{(i-1)} S_j^c
\end{equation}

A consequence to this definition is that outliers have a potential to drastically change the quality of the clustering outcomes \cite{chawla2013k}. To remedy this and the stochastic nature of the initialization process, the method is run several times on the same dataset, yielding the optimal solution from those runs. This does not guarantee the best solution for the problem, but the solution is approximated. The problem \textit{K-means} clustering attempts to solve is proven to be NP-hard \cite{chawla2013k, mahajan2009planar} but the algorithm itself has a time complexity of $O(n^2)$ \cite{pakhira2014linear}.


\subsection{Hierarchical clustering}

Hierarchical clustering is a deterministic clustering method. Each cluster formed is based on the entire dataset, in contrast to \textit{K-means} which approximates clusters by performing small changes to the cluster centers. The method produces clusters by iteratively combining the closest clusters according to the given linkage criterion (defined in section 2.2.2.2). The two primary strategies for forming clusters are agglomerative and divisive. Agglomerative clustering initializes one cluster for each data point and combines them in a hierarchy according to the linkage criterion until all clusters are part of the hierarchy. Divisive strategies initializes one universal cluster for all data points and proceeds to separate the points into distinct clusters according to the linkage criterion. The method proceeds until all data points are separated to their own cluster within the unifying hierarchy. The project described in this thesis uses the agglomerative strategy. All strategies rely on specific distance metrics and linkage criteria \cite{murtagh1983survey}.

\subsubsection{Distance metrics}
Let $u$ and $v$ be vectors of the same dimension $n \in \mathbb{N}$. The \textit{Euclidean distance} (also called \textit{L2-distance}) metric can be used to measure distance between the vectors in Euclidean space. The \textit{Euclidean distance} between $u$ and $v$ is defined by formula (\ref{eqn:euclid}).

\begin{equation}
\label{eqn:euclid}
d(u, v) = \sqrt{\sum_{i=1}^n (u_i - v_i)^2} 
\end{equation}

The \textit{Manhattan distance} (also called \textit{L1-distance}) metric is also a viable alternative, if the distance is to be measured in blocks. The distance is akin to finding a shortest path among blocks and is therefore calculated as expressed in formula (\ref{eqn:manhattan}).

\begin{equation}
\label{eqn:manhattan}
d(u, v) = \sum_{i=1}^n |u_i - v_i|
\end{equation}

\textit{Cosine similarity} measures similarity between vector angles and suits situations where certain vectors are expected to be similar. Should the vectors be sizable in terms of dimensionality, this method will yield varying results, especially if the elements have vary significantly in each dimension. It is calculated as expressed in formula (\ref{eqn:cosine}).

\begin{equation}
\label{eqn:cosine}
d(u, v) = \frac{\sum_{i=1}^n u_iv_i}{\sqrt{\sum_{i=1}^n u_i^2}\sqrt{\sum_{i=1}^n v_i^2}}
\end{equation}

\subsubsection{Linkage Criteria}
In order to measure distance between clusters it is essential to know between which points the distance should be measured, since clusters often consist of several points. Linkage criteria describe the method for determining how the distance metric will be applied. In this project, we use the library SKlearn and the already defined methods within it to perform our analyses; the following criteria are therefore the only focus for this subsection. SKlearn defines four criteria in the documentation: Single linkage, complete linkage, average linkage and ward linkage \cite{scikit}. Depending on which criterion is applied, the results may differ considerably.

Single linkage goes through each pair of clusters measuring the distance among all points within one with respect to the other. The distance between these clusters is determined to be the distance between the two closest points. Let $U$ be the elements in the first cluster and $V$ be the elements of the second. The distance between the first and the second cluster is defined formally in formula (\ref{eqn:single}).

\begin{equation}
\label{eqn:single}
d(U, V) = \forall_{u, v \in U, V} min(d(u, v))
\end{equation}

Single linkage tends to produce trivial results, forging a hierarchy in a chain where individual elements slowly merge with the bigger cluster. In contrast, complete linkage considers the largest distance between two points for every pair of clusters. The distance between two clusters then becomes the distance between the points which are the furthest apart, formally expressed in formula (\ref{eqn:complete}).

\begin{equation}
\label{eqn:complete}
d(U, V) = \forall_{u, v \in U, V} max(d(u, v))
\end{equation}

By considering the largest possible distance between two clusters, this criterion bypasses the setback of single linkage, allowing more clusters to form before merging into one unifying cluster.

Average linkage calculates the average between all elements for every pair of clusters and merges the ones possessing minimal average distance. It is formally described by formula (\ref{eqn:average}).

\begin{equation}
\label{eqn:average}
d(U, V) = \frac{1}{|U||V|}\sum\limits_{u\in U} \sum\limits_{v\in V}  d(u, v)
\end{equation}

Ward linkage represents distance by how much the summed square would increase by merging them. The method aims to merge the clusters such that the within cluster variance is minimal. Let $c_a$ be the center of cluster \textit{a}, then ward linkage is expressed formally by formula (\ref{eqn:ward}) \cite{shalizi2009distances}.


\begin{equation}
\label{eqn:ward}
d(U, V) = \frac{|U||V|}{|U|+|V|}||c_U - c_V||^2
\end{equation}



\subsection{Feature Selection}

In many cases, the data available contains numerous features; e.g. different frequencies on a spectrum, which often helps to build sufficient classifiers, as the model may find non-trivial patterns among the features. To avoid expanding the dependence on large datasets and to minimize the computation time, it is often necessary to rid the data of certain features.  
Ideally, the features selected for removal are those which provide the least information or are completely uncorrelated with the subject under study. An example for such a feature would be the color of someones clothes correlated with the chances of said person seeing a squirrel on that day. In other words, features are removed if they possess minimal correlation to other features or lack correlation entirely \cite{dash1997feature}. Features that possess the necessary expressive information are not always trivial, and there are several ways in which they may be found. In this project we exclusively use one form of feature selection with the SKlearn library. The SelectKBest method is a method which ranks features by their significance according to some scoring function. In this project, we use the f-classif method to score the features in the data set. The method computes the F-value using ANOVA for each feature in the data provided, the features are then sorted according to the F-value after which, SelectKBest returns the $k$ features with the highest score.


\section{Deep Learning}

Deep Learning (DL) is part of machine learning and concerns the use of massive models. DL is commonly used in association with Artificial Neural Networks (or simply Neural Networks) which have been used to great success in classification and regression tasks alike. In this section, we review the preliminary methods central to Neural Networks in the context of DL. The concepts of activation functions, layers and optimizers are covered in context of what the project requires. 

\subsection{Neural Networks}
Neural Networks are machine learning models which have been used to great success during the $21$st century; in no small part due to the increase in computational power over the past decade. With the use of Neural Networks, several fields including Natural Language Processing, Encoding and Image classification have undergone revolutionary leaps in performance regarding optimization due to the predictive power of these networks \cite{sharir2020cost, kukacka2012overview, zhang2015deep, lee2017deep}. At the same time they are heavily criticized for their complexity, yielding a structure much more akin to a so-called \textit{"black box"} than a reliable and deterministic method for prediction. This complexity is due to numerous different structural typologies available at present and an awesome number of learnable parameters \cite{qiu2004opening}. A consequence of this is hard skepticism regarding the correctness of their functionality within practical use. While these models have shown great promise when compared to their human counterparts, the question remains whether or not perfect performance can be yielded from the constructed models.

A Neural Network consists of a set of learnable parameters $W_n$, $n = 1,...,k$ for a model possessing $k$ layers. These parameters are commonly referred to as weights and are matrices with arbitrary dimensionality, with a set number of parameters for each element in the input $x$. The first set of weights have one dimension set to the shape of $x$ and the other shapes are chosen according to the size of the layers specified by the user. The last set of weights $W_k$ has the size of the expected output signal i.e. the number of different categories available in the context of classification. The layers denote the size of the different shapes the input is transformed into as the input propagates through the architecture. The input is propagated through the architecture via the dot product of the weights and the layer signals, yielding a new vector of shape $l_n$. Layers can also be convolutional, meaning they are multi-dimensional structures which can be used in the context of image classification and Natural Language Processing, where input can be read in sequences, rather than giant data structures. This is managed by initializing smaller kernels which are able to compute signals from the input by only observing the defined size, they move over the entire input by steps called strides after which a pooling layer is used to summarize the final layer signal. Between each transformation, an activation function is used to transform the signal further in a non-linear fashion. Each layer transformation $f_n$ is then the yielded signal from the activation function given the dot product between the current and preceding layer. The signal of layer $l_i$ where $ 1 < i \leq k $ is then the result of the activation function $\sigma$ of the dot product of the preceding layer signal $l_{i-1}$ and the weights $W_i$. This layer function is denoted by the output of the nested function call of all functions $f_0, f_1, ... f_i$ of input $x$ as expressed in formula (\ref{eqn:prop}) \cite{wang2003artificial}.

\begin{equation}
\label{eqn:prop}
f_i(f_{i-1}(...f_1(x))) = l_i = \sigma(W_i \cdot l_{i-1} + b_i)
\end{equation}

The variable $b_i$ is the bias term for the activation. Its inclusion allows the activation curve to be moved along the x-axis. This shift in position of the activation allows the model to further change its own behavior through the learning process. It avoids bias towards the y-intercept of the activation function.  

\subsection{Activation functions}

Activation functions are used in neural networks to transform the input in a non-linear fashion. The function can be any function on numerical elements, the only requirement is that it must be derivable for all possible inputs. The functions usually transform the signal to be in a certain interval such as the hyperbolic tangent function (tanh) or the sigmoid function $\sigma$. The sigmoid function was originally used due to the similarities with the activation of biological neurons. The "s"-shaped curve of the function transforms any signal to the interval $[0, 1]$. It is comparable to tanh, which transforms signals into the interval $[-1, 1]$. The functions are formally expressed in formula (\ref{eqn:sig}) and (\ref{eqn:tanh}), respectively.

\begin{equation}
\label{eqn:sig}
\sigma(x) = \frac{1}{1+e^{-x}}
\end{equation}

\begin{equation}
\label{eqn:tanh}
tanh(x) = \frac{e^x - e^{-x}}{e^x + e^{-x}}
\end{equation}

Choosing an activation function depends on what range the user wants the signals to fall into. Sigmoid and tanh work sufficiently well for many models, giving promising results for many different tasks within DL. One flaw is that they are computationally expensive to calculate. The rectified linear unit (ReLU) is an activation function which is easily computed for many elements without the necessity for significant amounts of computational resources. The ReLU function returns the input itself if the input is greater than zero, and zero otherwise. The derivative of ReLU is similarly efficient to compute, the derivative is one for input greater than zero, and zero otherwise. The function is shown to outperform sigmoid and tanh as activation function in many cases, which has promoted its use in several applications. The drawback of ReLU is the derivative of 0 for signals of zero and below. The derivative is used during the training process to introduce changes to the model. With ReLU, the activation signal will become zero if the signal is smaller than zero. The neurons which suffer from this problem are referred to as "dead", since the weights used for their activation always bring the signal to or below zero. A possible fix for this is the leakyReLU function, which introduces a slight slope to the ReLU curve for values below zero. The ReLU function may then be defined through the leakyReLU function with a slope of zero for signals below zero. The derivative of leakyReLU then becomes one for values greater than zero, and a small real number for values below zero \cite{agarap2018deep, lu2019dying}.

The final activation function introduced in this section is the softmax function. Softmax is a method which transforms the signal in the output into a probability distribution i.e. scales the signals according to the maximum element and transforms the data to have a sum of one. This is done using Euler's constant $e$ as base, dividing $e$ raised to the power of each element in the input signal by the sum of all exponents. This is formally expressed in formula (\ref{eqn:softmax}) \cite{misra2019mish}.

\begin{equation}
\label{eqn:softmax}
\sigma(x)_i = \frac{e^{x_i}}{\sum_{j=1}^k e^{x_j}}
\end{equation}

\subsection{Regularization}
Regularization methods are used during the training process to aid in generalization for Neural Networks. One such method is dropout, which assigns a dropout rate to specifically selected layers. Dropout randomly reduces signals of individual neurons in the selected layers to zero which prevents the model from enforcing connections which become heavily affiliated with certain types of predictions. This is especially important in large architectures where layers can consist of hundreds of neurons where strong reinforcements are easily established \cite{srivastava2013improving}. Gaussian noise may also be added to all signals in any layer to shift the signals in them sporadically. However, this method requires some knowledge about the range of the signals within the layers, as large additive noise can remove any necessary information from the input which affects the gradient significantly. The model will learn to reduce dependency on noise during training, provided the noise is not "destructive". For example, adding Gaussian noise drawn from a normal distribution with a standard deviation of five to signals returned by the sigmoid activation function. This will shift the distribution of signals which removes valuable signal information. Batch normalization is a regularization technique which normalizes the change applied to $W_n$ during training over several batches of input and helps  regularizing the model \cite{santurkar2018does}.

\subsection{Optimization}

Neural Networks have many usable loss functions depending on context. In the context of classification with multiple categories, categorical cross-entropy is commonly used to measure error between the prediction of the network (usually produced with softmax or sigmoid activations) when the prediction is meant to categorize the input. The cross-entropy loss is calculated as the sum of negative elements of the true label $y^t$ multiplied by the logarithm of the predicted label $y$. Let $k$ be the number of elements in the output signal $y$, the cross-entropy loss is formally expressed in equation (\ref{eqn:crossent}).

\begin{equation}
\label{eqn:crossent}
-\sum_{i=1}^k y_{i}^t log(y_i)
\end{equation}

Equation (\ref{eqn:crossent}) measures the entropy between two distributions, entropy being the loss of information between two different distributions. The equation assumes the values within $y$ are elements of a probability distribution, i.e. the sum of the elements within $y$ equals one. The negation of the logarithm is used to bring the elements of the sum to a positive scope. This ensures the loss will be positive, since all elements in $y$ and $y^t$ are in the interval $[0, 1]$. The sigmoid function may also be used in association with cross-entropy, as the values will be transformed into the zero-to-one interval. Other activation functions such as tanh and ReLU run the risk of bringing elements in the sum to the negative scope or undefined (as the logarithm is undefined for values less than zero) \cite{krippendorff2009mathematical, shannon2001mathematical}.

The computed loss between the data labels and the predicted labels is then used by the optimizer to change the learnable parameters. The backpropagation algorithm calculates the partial derivatives of the computed loss with respect to each learnable parameter. The collection of the derived parameters are called the gradient. The gradient is scaled by the learning rate (usually a value less than $0.01$) defined in the optimizer, this allows for small changes to each parameter which allows the model to approach the minimum of the loss function at a speed proportional to the learning rate. Using the gradient in this way is called gradient descent and it is the common method of learning all optimizers use. Adam is an optimizer introduced by Kingma and Lei Ba \cite{kingma2014adam} which has proven to be efficient in contrast to other optimization methods such as Gradient Descent, AdaGrad and RMSprop. Adam uses an adaptive learning rate to approach the minimum of the loss function. Adam requires four different parameters for the algorithm to run. These are the learning rate $\alpha$, stochastic decay rates $\beta_1$ and $\beta_2$ and a small constant $\epsilon$ used to avoid division by zero for the update equation. The algorithm described by Kingma and Lei Ba also requires two vectors $m$ and $v$ used to describe the "moment" of the gradient and the squared gradient respectively (initialized as zero vectors). The gradient update at timestep $t$ is expressed by the following formulas:

\begin{center}


$t = t + 1$

$g_t = \nabla L_t(W_{t-1})$

$m_t = \beta_1 \cdot m_{t-1} + (1-\beta_1) \cdot g_t $

$v_t = \beta_2 \cdot v_{t-1} + (1-\beta_2) \cdot g_t^2 $

$m_t' = \frac{m_t}{1-\beta_1^t}$

$v_t' = \frac{v_t}{1-\beta_2^t}$

$W_t = W_t - \alpha \cdot \frac{m_t'}{\sqrt{v_t'} + \epsilon}$

\end{center}

Each vector at timestep $t$ uses the values for the "moments" at the previous timestep $t-1$. Each "moment" is then updated using the previous timestep and the stochastic decay rates. Each "moment" vector is normalized by element-wise division of the "moment" vectors and their respective decay rates to the power of $t$. $\beta_1$ and $\beta_2$ are initialized to be $0.9$ and $0.999$ respectively, this ensures that subtraction by one will maintain the relative scope between the "moment" vectors and their corrected counterparts. The parameters themselves are then updated by subtracting the current parameters by alpha multiplied by the decay rate (calculated by the fraction of the "moment" vectors with the $\epsilon$ parameter added to the denominator). The "moment" vectors then give each feature a unique learning rate which accelerates training. 

\chapter{Data Exploration}
Deep learning models require tremendous amounts of data, to ensure the model works well the data must also possess sufficient characteristics to approximate the sample population from which it was extracted. To satisfy this requirement we examine the data in attempt to remove outliers and determine whether the data is sufficient for classification. Moreover, certain tumors may be heterogeneous \cite{friedmann2014glioblastoma}, which may be problematic for a classifier as heterogeneous samples lack in shared characteristics. In this chapter the data available to the project is examined in greater detail; details for how the Raman spectra were prepared is given to document the preprocessing of spectra for future use. First we describe the mathematical representation of the samples. Since the number of samples is too small to use in a deep learning model, we explain how each sample may be separated into individual spectra; this separation yields a drastic increase in the number of available training examples. We then explain how to balance the data; an unbalanced dataset would likely introduce bias in the deep learning model, rendering its desired predictive capabilities uncertain. We achieve this by duplicating underrepresented samples-classes in the dataset. Furthermore, this balancing is performed to maintain majority and minority classes, thus retaining some distributional information from the original dataset. The quintessential purpose of this chapter is to analyze the data using k-means clustering and hierarchical clustering for detecting non-tumor spectra or otherwise erroneous spectra. These methods are tested in contrast to other outlier detection techniques such as the standard deviation test and the interquartile range method. Adrian's criterion is presented and compared to the the spectral images extracted from the samples.
Another point of interest in this project is the identification of representative frequencies within the spectra. Each spectra belongs to a tumor which can be categorized by six different classes. The hypothesis states certain frequencies should be sufficient in determining which class the tumor belongs to. For this feature selection is used, representing each frequency within the spectra as distinct features. This task is simplified by the new representation of the data separated into lists of spectra rather than a collection of spectra represented by the tumor. However in order to extract such features the data must be devoid of outliers. Should outliers exist within the dataset, the features given by the methods used will be influenced and may yield conflicting results with the ground truth. To prepare for this the data is plotted for visual inspection using various methods to be discussed in the following section on feature selection. It is confirmed by the provider that the majority of samples include faulty spectra e.g. spectra of blood drops on the sample or plastic which may be reflected form underneath thin tissue. Furthermore some tissue may be necrotic which will affect the spectral signal. Using the extracted features a model can potentially form around the data faster which can be essential as deep learning training require significant computing resources. The features are extracted before and after the removal of problematic spectra for comparison.


\section{Data Representation}
The data consists of the Raman-spectra extracted from the tissue of glioma tumors from 45 patients. Multiple samples of tissue were extracted from the same patient in some cases, yielding several samples for the respective patient. To maintain separation among the patients, the samples are sorted by their respective patient of origin. This is necessary, since there is uncertainty regarding the homogeneity among patient samples. The data will be separated into three separate datasets. These sets are called training set, validation set and testing set. All datasets will consist of unique patients to avoid scenarios in which the model overfits to a patients tumor sample and as a result of heterogeneity. This structure also allows for easier handling of the number of patients in the sample-classes, allowing for analysis on each class exclusively. 

There is also large variation with regard to the sample shape within the data. Each sample is a 3-dimensional array of shape $(w, h, 1738)$ where $w$ and $h$ are the width and height of the sample, respectively. This formalization is necessary, as width and height have non-zero variance among different samples. The shape is a result of how the tissue was scanned. In each case the tissue was placed inside the instrument and scanned successively from side to side. This makes it possible to display each sample as an image, by substituting the third dimension (denoted above by $1738$) a color value denoted by which class the spectra belongs to. The number $1738$ is constant through all samples and represents the length of a modified Raman-spectra which is performed by the provider, each element a unique frequency. Furthermore each element inside these arrays is a real number without clear bounds. The largest absolute element found within the complete dataset is $79427.0625$, some values are negative which is confirmed by the providers to have significance for the projects purpose. The project aims to predict which subdivision the spectra belong to by feeding in one of these samples, i.e., one vector of shape $(1, 1738)$. This strategy is inspired by Liu et al.\cite{liu2017deep}, who managed to get satisfactory performance by training a model on raw spectra. This representation is of great interest, since the value of each spectrum is independent from the surrounding spectra. Separation at this level yields a dataset with more than $300,000$ datapoints, which better suits deep learning tasks.

 
To prepare for the project each samples spectra is collected and plotted in one single plot to compare spectral information from the sample itself. An example of such plots can be seen in Appendix \ref{appendix:spectraplot}. Patient HF-1887 is removed from the project completely due to the skewed baseline in the spectra.


\section{Data preparation}
In this section we explain our qualitative analysis on the data, done to determine the plausibility of our model. Each sample is categorized according to their subdivision; there are six distinct subdivisions as defined by Ceccarelli et al. \cite{cellsubsets}, denoted LGm1 - 6. As an initial step each samples spectra is analyzed by visual inspection. The spectra are plotted on a two dimensional surface as lines, each line a unique spectrum. Through this analysis one sample is discarded due to the tilted baseline of the spectra, none of the other samples share this problem. A sample which shares a general shape with the other samples is shown in contrast to the sample selected for removal in \ref{appendix:spectraplot}. Another sample shows a concerning number of spikes in contrast to the other samples. The number of spectra is also considerably larger compared to the others, which is limiting for some of the methods selected for the analysis. For this reason the sample is also discarded.

\subsection{Organization and Balance}
The model will as a consequence of it's learning-algorithm become biased towards certain predictions. This because as the model encounters frequent examples of a certain class, the connections which produce such predictions will strengthen. Over exposure to examples of a certain class will force the model to associate features with that class, redirecting focus from classes for which that feature could be significant. The initial data suffers heavily from this problem, a is shown in Table \ref{table:1}.

\newpage

\begin{table}[h!]
\centering
 \begin{tabular}{||c c c c c c c||} 
 \hline
 Class & LGm1 & LGm2 & LGm3 & LGm4 & LGm5 & LGm6 \\ [0.5ex] 
 \hline\hline
 \# of samples & 5& 11 & 4 & 10 & 11 & 4 \\ 
 \hline
 \# of spectra & 37319 & 71846 & 31931 & 50660 & 62256 & 20176 \\
 \hline
 percentage & 14\%& 26\% & 12\% & 18\% & 23\% & 7\% \\
 \hline

\end{tabular}
\caption{Table showing the distribution of data in the initial dataset after removing the problematic samples. The number of samples are displayed on the first row, the number of spectra in each class is shown on the second row. The percentage of the entire dataset is shown on the third row. The majority class is LGm 2 and minority is LGm 6.  Classes LGm 1, 3 and 6 must be expanded to balance the data.}
\label{table:1}
\end{table}

Table \ref{table:1} shows the per class separation in the data, the header row shows the labels of each class. The first row shows the number of samples belonging to each class, these are the tumors which will be analyzed. The second row displays the total number of spectra across each class; these must be considered for balancing. Note the equal amount of samples in LGm3 and LGm6, but the difference in number of spectra within them. This is due to the varying size of all samples drawn from the tumors. Some samples share the same size, however the important fact is that the samples lack a uniform shape, which must be considered during the analysis. The last row shows the percentage each class makes of the entire dataset. Initially LGm2 is the majority class while LGm6 is the minority, consisting of only $7$\% of the entire dataset.

Before the data is balanced, the testing data is selected and separated from the rest manually. This is done by separating at least one patient and all their samples from the rest of the data. This way it will be possible to test if the model is develops bias towards the patients in training and if the patient samples are heterogeneous with respect to the other samples of the same class. The test-samples are chosen manually, samples are chosen with the criterion that approximately $30\%$ of each class is represented in the test set. Balancing the classes which contain less elements by a factor larger than or equal to two compared to the majority class (LGm2) is done by repeating the spectrum in each sample by that factor. Following this method the majority class will stay the majority which can be crucial provided the sample pattern is similar to the set of all other unseen samples. The resulting dataset is gained by doubling the samples in LGm1, tripling the samples in LGm3 and quadroupling the samples in LGm6. The distributions of the training dataset is shown in Table \ref{table:2}.
\\
\\
\begin{table}[htb]
\centering
 \begin{tabular}{||c c c c c c c||} 
 \hline
 Class & LGm1 & LGm2 & LGm3 & LGm4 & LGm5 & LGm6 \\ [0.5ex] 
 \hline\hline
 \# train & 21289	& 51698	& 20635	& 34276	& 37492	& 12976 \\
 \hline 
 \# test & 14945 & 20140 & 11296 & 16384 & 24764 & 7200 \\
 \hline

\end{tabular}
\caption{Distribution of the training data and the testing data}
\label{table:2}
\end{table}

Table \ref{table:3} shows the distribution of the training data and testing-data. The training data is then balanced exclusively. This is not required in the testing data, since it will have no effect on how the model is developed through training. The training data is balanced by replicating each spectra in every patient of the classes which are under-represented. The resulting training-set is gained by doubling the spectra in LGm 1 and LGM 3 and  the spectra in LGm 6. The final distribution of the training data following this procedure is shown in table \ref{table:3}

\begin{table}[htb]
\centering
 \begin{tabular}{||c c c c c c c||} 
 \hline
 Class & LGm1 & LGm2 & LGm3 & LGm4 & LGm5 & LGm6 \\ [0.5ex] 
 \hline\hline
 \# train & 42578 & 51698 & 41270 & 34276 & 37492 & 38928 \\
 \hline 

\end{tabular}
\caption{Distribution of the testing data following balancing}
\label{table:3}
\end{table}

\section{Analysis}

Following the balancing, the first step in the analysis is to find the frequencies which best describe the data with respect to the methylation-types. Each number in the spectra is a frequency at which the scattered light is gathered. This light is expected to be sufficient for predicting the methylation-type of the tumor-tissue. It is speculated that to sufficiently categorize the spectra into the methylation-types only certain frequencies are required. For this reason the best features are extracted with SelectKBestfeatures \cite{scikit} which is given the f-classif method for ranking the features which yields features deemed significant by ANOVA. The 70 best features were extracted from the training-data in which there are spectra which originate from non-tumor tissue. The features are displayed in Appendix \ref{appendix:features0}. Before the features are extracted, each spectra is standardized using z-score standardization, to give each spectrum a mean of zero and a standard deviation of one. This is done for ease of comparison among the spectra. The extracted features show that regions of interest do exist on the spectra. This can be seen by the integers which have a difference of one, suggesting that the region of interest exist somewhere in specific parts of the spectra. It is worth noting here that the features selected might be correct provided the amount of non-tumor spectra is sufficiently small to be ignored by the feature selection method. Due to this uncertainty, the data will be separated from the outliers and feature selection will be performed a second time.

To avoid bias the analysis is performed on the training data exclusively. The goal of the analysis is to find a uniform criterion which each spectrum must fulfill to be considered \textit{clean}. Spectra which fail to satisfy this criterion will be discarded form the project entirely. In this section the methods of analysis used are described and their results examined, the section begins by examining the standard deviation test and interquartile range method, these are deterministic methods that rely solely on the values found within the data. K-means clustering and agglomerative clustering are then performed on the data in attempt to capture potentially complex patterns within the data. The section ends by presenting Adrians criterion, which is a criterion for finding problematic spectra defined by the data provider.

\subsection{The standard deviation test}

The standard deviation test is a test by which the data is centered around the mean and given a standard deviation of one. With this setup outliers are defined as points which are separated from the mean by 3 standard deviations or more. We measure the mean and standard deviation on each frequency from the unbalanced training-set; the values are then used to standardize the spectra belonging to each tumor. A spectrum is deemed to be an outlier if the number of frequencies in that spectrum exceed an arbitrary value. We approximated the value by performing the test once while monitoring the average number of frequencies which lie $3$ or more standard deviations from the mean. In any given sample, each spectrum includes on average $111$ frequencies which are separated by 3 standard deviations or more from the mean. We specify that a spectrum whose number of frequencies which fail the test is a spectrum possessing more than $111$ outlier-freuencies, is deemed an outlier. A display of which spectra would be removed from the samples of type LGm1 is shown in Appendix \ref{appendix:stdTest}. Using this test, many samples outline small spots on each sample, it suggests there is something present in those places, but they do not possess a clear shape by which we can decide whether to include them or not. Some samples do show clear areas of outliers but areas within those regions are falsely classified as non-outliers. There is also one sample for which the test classifies approximately $97 \%$ as outliers which is not confirmed by the provider to be problematic. Despite promising results for some samples, we decide to turn to other methods for outlier detection.

\subsection{The Interquartile range method}

Similar to the standard deviation test the interquartile range method is a purely statistical analysis method which detect outliers in terms of which percentile the points fall into. The 25th and 75th percentiles are calculated on each frequency for the entire training set. Unlike the previous test, this method yields a varying amount of outliers for each sample. We instead define the allowed number of outlier frequencies within one spectra to be equal to the average number of outlier frequencies within the analyzed sample. The resulting outliers are shown in Appendix \ref{appendix:iqrMethod}. Compared to the standard deviation test, this method produces similar results. However, many regions are better represented by the interquartile range, showing well defined areas where unknown material is clearly present. The amount of individual spots are less frequent which shows promise in the method, as e.g. blood is expected to cover a larger area if present.

\subsection{Hierarchical clustering}

The next method of analysis for outliers is hierarchical clustering, we choose to utilize agglomerative clustering as an arbitrary choice. Due to the algorithms demanding time complexity and memory constraints, we choose to analyze each sample separately to avoid these issues. This means there is little to gain in comparison among separate samples however the deterministic nature of the algorithm shows promise for rigorous results in the clusters. We compare the results of different distance metrics and choose to utilize Euclidean distance to measure distance among the clusters. This conclusion is due to the uncertainty in vector shape and angles on which Cosine similarity is dependent. The Manhattan distance would be a better choice compared to Cosine similarity, however, Euclidean distance magnifies long distances which should aid the algorithm in selecting clusters for agglomerative merging. Following choice of distance metric we examine the linkage criteria available. In each case, the algorithm is set to run multiple tests where it separates the data into different amounts of clusters. We choose to compute between $2$ and $7$ clusters.

Single linkage merges clusters by way of merging those which posses points with minimal distance. One flaw in this criterion is that clusters may easily be merged from one parent cluster until all points have been connected in a chain. The criterion yields clusters which appear as individual points in the spectra and fail to detect areas where known outliers are present in the samples, suggesting the aforementioned flaw is present in this methodology. This is especially apparent if we allow the method to use more than two clusters. All different cluster amounts fail to separate the outliers and instead separate a small number of points. Examples of these cluster results are displayed in \textbf{Appendix}.

Average linkage yield similar results as single linkage with the exception that some areas become apparent as we increase the number of clusters. Moreover, the criterion does not have a set number of clusters which is guaranteed to include all outliers. For certain samples the outliers are visible when forcing the algorithm to agglomerate to two clusters and others only show them once five or more clusters are allowed. It is possible to use the criterion for discarding the outliers if the majority cluster is preserved when computing 7 clusters while the rest of the spectra are discarded, but this would not remove all outliers and some problematic samples in LGm3 would have the majority of outliers preserved. 

Complete linkage merges the clusters which posses elements with the smallest possible maximal distance between them. This avoids the setback of single linkage as a majority cluster is harder to form early under the criterion. The method produces similar results as average linkage, few areas with outliers are detected when fewer clusters are permitted. However, some outliers are present when computing 2 or 3 clusters. Allowing 7 clusters to form allows the algorithm to capture many of the areas of interest, however, this captures too much in some samples. We recommend the majority cluster is maintained when allowing 4 clusters with the method. But this suffers from the same setback as our conclusion on average linkage.

Ward linkage merges clusters which posses minimal variance between their respective elements and as such works well with Euclidean distance. We do stress that the clusters are merged by measuring variance among cluster elements and there is no guarantee the outlier spectra should share in characteristics which would result in low inter-cluster variance. Despite this lack of guarantee the clusters form at the precise location of outliers. Allowing 7 clusters produce a near picture perfect image of the biological tissue from which the spectra were measured. These results show that the algorithm is capable of organizing the spectra according to their visual information which aid us in understanding shape and state of the samples. Some of the spectral images fromed by the clusters are shown in \textbf{Appendix}. The issue is finding a uniform criterion on which we can discard the outliers. Removing every cluster except for the majority cluster in the case where 7 clusters have been formed would remove legitimate spectra which are suitable for training a model. In fact, there is no optimal choice in this case, as certain samples have their outliers sufficiently captured in a setup allowing for 2 clusters. While others show their outliers in arbitrary numbers of clusters. Selecting too many clusters will result in some samples loosing legitimate spectra which is undesirable in context of maintaining a sizable dataset. By visual inspection, we deem the optimal choise to be 3 clusters since many outliers are present in this choice, but the problem is still present in this choice.

\subsection{K-means clustering}


\subsection{Adrians criterion}

Adrians criterion is a criterion specified by Adrian Lita for separating outlier spectra from the rest. It states that should any value between frequency $1463$ and $1473$ be below $5000$, then that spectrum is defined to be and outlier. Given that this criterion is defined by the provider, and the lack of consistency in the other methods explained in this section, we choose to discard outliers found by adrians criterion despite the criterion missing certain areas such as the upper part of sample HF-1293 which does satisfy the criterion, but is also confirmed to be plastic. Our hope is that the model will be able to ignore these and other potential outliers which escape Adrians criterion.


Concluding this section we also perform clustering with the selected features gained from the beginning of this section \textbf{Reffer to appendix for this!}. Limiting the spectra to the features which best separate them into the 6 LGm-classes has advantage for this algorithm, many areas of outliers become easily spotted by allowing 2 clusters in many cases. Complete linkage gains the biggest advantage out of this method. None of the above mentioned configurations manage to capture the upper part of HF-1293, however, when the data is limited to the features selected, the area becomes visible to some extent. This suggests there are frequencies in the spectra which complicate the separation of outliers. While we see that the method clearly works in many cases, using all features proves to be troublesome for the majority of linkage criteria. Furthermore, the features are computed by means of separating the spectra into the 6 LGm-classes, they are not computed due to their efficiency for detecting legitimate spectra and outlier spectra. Following the removal of outlier spectra we decide to label each spectrum by a binary value, a spectrum is assigned true if it is legitimate and false if it is labeled an outlier by Adrians criterion. We then run feature selection once more, extracting the features which bets separate legitimate spectra from outliers. The features are displayed in Appendix \textbf{appendix:Features}.

\chapter{Results}

\chapter{Conclusion}

\bibliographystyle{ieeetr}
\bibliography{ref}% ref.bib


\begin{appendices}
\chapter{Spectral Plots}

\begin{figure}[h]
\label{appendix:spectraplot}
    \centering
    \subfloat[\centering Spectra from patient HF-1293. The rest of the samples available share in this pattern with some deviations]{{\includegraphics[width=5cm]{images/1293graph.JPG} }}
    \qquad
    \subfloat[\centering Spectra from patient HF-1887. The frequencies tilt towards the upper part of the plot. The example is decidedly removed from the analysis.]{{\includegraphics[width=5cm]{images/1887graph.JPG} }}%
    \caption{Examples of samples drawn from the data, HF-1293 display a common pattern across all samples, HF-1887 is removed due to problematic handling}%
\end{figure}

\chapter{Feature Selection}

\label{appendix:features0}

% No standardization!
509,  521,  522,  523,  524,  525,  526,  527,  528,  529,  530,
        532,  533,  538,  539,  540,  541,  545,  546,  547,  548,  549,
        550,  551,  552,  553,  562,  563,  647, 1449, 1450, 1451, 1452,
       1453, 1454, 1455, 1456, 1457, 1458, 1459, 1460, 1461, 1462, 1463,
       1464, 1465, 1468, 1469, 1470, 1471, 1472, 1473, 1474, 1475, 1476,
       1477, 1478, 1479, 1480, 1481, 1482, 1483, 1484, 1485, 1487, 1492,
       1494, 1495, 1496, 1497

\chapter{Spectral Images For Outlier Detection}

\label{appendix:HF1293Comparison}

\begin{figure}[H]

    \centering
{\includegraphics[width=15cm]{images/Ward_linkage/LGm-1/HF-1293_13_1.h5_1.png} }
\caption{Single linkage on sample HF-868 from LGm1, The test fails to detect any outlier areas. LEftmost image is the result of the model computing 2 clusters. The number of clusters increase towards the rightmost image.\label{fig:SL_HF868}}%

\end{figure}


\begin{figure}[H]

    \centering
{\includegraphics[width=15cm]{images/KMeans_full/LGm-1/HF-1293_13_1.h5_1.png} }
\caption{Single linkage on sample HF-868 from LGm1, The test fails to detect any outlier areas. LEftmost image is the result of the model computing 2 clusters. The number of clusters increase towards the rightmost image.\label{fig:SL_HF868}}%

\end{figure}


\end{appendices}

% Multiline command from stackexchange, hide this text
\newcommand{\comment}[1]{}
\comment{
\appendix
%\chapter{Appendix}

\begin{figure}[h]
\label{appendix:spectraplot}
    \centering
    \subfloat[\centering Spectra from patient HF-1293. The rest of the samples available share in this pattern with some deviations]{{\includegraphics[width=5cm]{images/1293graph.JPG} }}
    \qquad
    \subfloat[\centering Spectra from patient HF-1887. The frequencies tilt towards the upper part of the plot. The example is decidedly removed from the analysis.]{{\includegraphics[width=5cm]{images/1887graph.JPG} }}%
    \caption{Examples of samples drawn from the data, HF-1293 display a common pattern across all samples, HF-1887 is removed due to problematic handling}%
\end{figure}


\label{appendix:features0}

(OLD FEATURES, done poorly on z-score standardized data)
begin 250  309  522  523  524  525  526  527  528  529  553  563  645  646
  647  648 1450 1451 1452 1453 1454 1455 1456 1457 1458 1459 1460 1461
 1462 1463 1464 1465 1469 1470 1471 1472 1473 1474 1475 1476 1477 1478
 1479 1480 1481 1482 1483 1484 1485 1486 1487 1488 1489 1490 1491 1492
 1493 1494 1495 1496 1497 1498 1499 1500 1501 1502 1503 1511 1512 1513 end

Adrians criterion concerns frequencies labeled by the features from 1463 to 1473. Almost all features are present, with the exception of features 1466, 1467 and 1468. 

\label{appendix:hierimg0}



}


\end{document}